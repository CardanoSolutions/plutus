\documentclass[a4paper]{article}

\title{Formal Specification of\\the Plutus Core Language (version 3.0)}
\date{\today}
\author{Plutus Team}

\renewcommand{\thefootnote}{\fnsymbol{footnote}}

% correct bad hyphenation here
\hyphenation{}

\usepackage{blindtext, graphicx}
\usepackage{url}
\usepackage{natbib}

% *** FONTS ***
%
\usepackage{amsmath}
\usepackage{amssymb}
\usepackage{stmaryrd}
\usepackage{stix}
\usepackage{alltt}
\usepackage{bussproofs}
\usepackage[mathscr]{euscript}

% *** FIGURES/ALIGNMENT ***
%
\usepackage{array}
\usepackage{float}  %% Try to improve placement of figures.  Doesn't work well with subcaption package.
\usepackage{subcaption}
\usepackage{caption}


\usepackage{subfiles}
\usepackage{geometry}
\usepackage{pdflscape}
\usepackage[title]{appendix}

\usepackage[T1]{fontenc}
\usepackage[dvipsnames]{xcolor}

\usepackage{listings}
\lstset{
  basicstyle=\ttfamily,
  columns=fullflexible,
  mathescape=true,
  escapeinside={|}{|}   %% Inside listings you can say things like |\textit{blah blah}|
}

\usepackage[colorlinks=true,linkcolor=MidnightBlue,citecolor=ForestGreen,urlcolor=Plum]{hyperref}
\usepackage{longtable}

% You're supposed to make this the final package
\usepackage[disable]{todonotes}
% \usepackage{todonotes}
\newcommand{\kwxm}[1]{\smallskip\todo[inline,color=yellow,author=kwxm,caption={}]{#1}\smallskip}
\renewcommand{\roman}[1]{\smallskip\todo[inline,color=green,author=effectfully,caption={}]{#1}\smallskip}

\newcommand{\T}[1]{\texttt{#1}}  %% This is used in some tables to make the Latex more compact

\newcommand{\termarg}{\smblkcircle}
\newcommand{\typearg}{\circ}  % stix redefines \bullet to something smaller

% % Stuff for splitting figures over page breaks
% \DeclareCaptionLabelFormat{continued}{#1~#2 (Continued)}
% \captionsetup[ContinuedFloat]{labelformat=continued}


%%% General Misc. Definitions

\newcommand{\red}[1]{\textcolor{red}{#1}}
\newcommand{\redfootnote}[1]{\red{\footnote{\red{#1}}}}
\newcommand{\blue}[1]{\textcolor{blue}{#1}}
\newcommand{\bluefootnote}[1]{\blue{\footnote{\blue{#1}}}}

\newcommand{\diffbox}[1]{\text{\colorbox{lightgray}{\(#1\)}}}
\newcommand{\judgmentdef}[2]{\fbox{#1}

\vspace{0.5em}

#2}

\newcommand{\hyphen}{\operatorname{-}}
\newcommand{\repetition}[1]{\overline{#1}}
\newcommand{\Fomega}{F$^{\omega}$}
\newcommand{\keyword}[1]{\texttt{#1}}
\newcommand{\inparens}[1]{\texttt{(} #1 \texttt{)}}
\newcommand{\construct}[1]{\texttt{(} #1 \texttt{)}}
% \newcommand\discharge[1]{\widehat{#1}}

%%% Term Grammar

\newcommand{\sig}[3]{[#1](#2)#3}
\newcommand{\constsig}[1]{#1}
\newcommand{\con}[2]{\inparens{\keyword{con} ~ #1 ~ #2}}
\newcommand{\abs}[3]{\inparens{\keyword{abs} ~ #1 ~ #2 ~ #3}}
\newcommand{\inst}[2]{\texttt{\{}#1 ~ #2\texttt{\}}}
\newcommand{\lam}[3]{\inparens{\keyword{lam} ~ #1 ~ #2 ~ #3}}
\newcommand{\app}[2]{\texttt{[} #1 ~ #2 \texttt{]}}
\newcommand{\iwrap}[3]{\inparens{\keyword{iwrap} ~ #1 ~ #2 ~ #3}}
\newcommand{\wrap}{\iwrap}
%% ^ Temporary fix to avoid substituting all occurrences of new keyword
\newcommand{\unwrap}[1]{\inparens{\keyword{unwrap} ~ #1}}
\newcommand{\builtin}[1]{\inparens{\keyword{builtin} ~ \mathit{#1}}}
\newcommand{\error}[1]{\inparens{\keyword{error} ~ #1}}

%% Extra untyped terms
\newcommand{\lamU}[2]{\inparens{\keyword{lam} ~ #1 ~ #2        }}
\newcommand{\appU}[2]{\texttt{[} #1 ~ #2 \texttt{]}}
\newcommand{\builtinappU}[3]{\inparens{\keyword{builtin} ~ #1 ~ #2 ~ #3}}
\newcommand{\errorU}{\inparens{\keyword{error}}}
\newcommand{\delay}[1]{\inparens{\keyword{delay} ~ #1}}
\newcommand{\force}[1]{\inparens{\keyword{force} ~ #1}}

\newcommand{\erase}[1]{\llbracket#1\rrbracket}

%%%  Type Grammar

\newcommand{\funT}[2]{\inparens{\keyword{fun} ~ #1 ~ #2}}
\newcommand{\ifixT}[2]{\inparens{\keyword{ifix} ~ #1 ~ #2}}
\newcommand{\fixT}{\ifixT}
%% ^ Temporary fix to avoid substituting all occurrences of new keyword
\newcommand{\allT}[3]{\inparens{\keyword{all} ~ #1 ~ #2 ~ #3}}
\newcommand{\conIntegerType}[1]{\keyword{integer}}
\newcommand{\conBytestringType}[1]{\keyword{bytestring}}
\newcommand{\conT}[1]{\inparens{\keyword{con} ~ #1}}
\newcommand{\lamT}[3]{\inparens{\keyword{lam} ~ #1 ~ #2 ~ #3}}
\newcommand{\appT}[2]{\texttt{[} #1 ~ #2 \texttt{]}}

\newcommand{\typeK}{\inparens{\keyword{type}}}
\newcommand{\funK}[2]{\inparens{\keyword{fun} ~ #1 ~ #2}}



%%% Program Grammar

\newcommand{\version}[2]{\inparens{\keyword{program} ~ #1 ~ #2}}

\newcommand{\evalbuiltin}[2]{\llbracket #1 ~ #2 \rrbracket}

%%% Judgments

\newcommand{\hypJ}[2]{#1 \vdash #2}
\newcommand{\ctxni}[2]{#1 \ni #2}
\newcommand{\validJ}[1]{#1 \ \operatorname{valid}}
\newcommand{\termJ}[2]{#1 : #2}
\newcommand{\typeJ}[2]{#1 :: #2}
\newcommand{\istermJ}[2]{#1 : #2}
\newcommand{\istypeJ}[2]{#1 :: #2}



%%% Contextual Normalization

\newcommand{\ctxsubst}[2]{#1\{#2\}}
\newcommand{\typeStep}[2]{#1 ~ \rightarrow_{ty} ~ #2}
\newcommand{\typeMultistep}[2]{#1 ~ \rightarrow_{ty}^{*} ~ #2}
\newcommand{\typeBoundedMultistep}[3]{#2 ~ \rightarrow_{ty}^{#1} ~ #3}
\newcommand{\step}[2]{#1 ~ \rightarrow ~ #2}
\newcommand{\normalform}[1]{\lfloor #1 \rfloor}
\newcommand{\subst}[3]{[#1/#2]#3}
\newcommand{\kindEqual}[2]{#1 =_{\mathit{k}} #2}
\newcommand{\typeEqual}[2]{#1 =_{\mathit{ty}} #2}
\newcommand{\typeEquiv}[2]{#1 \equiv_{\mathit{ty}} #2}


\newcommand{\inConTFrame}[1]{\conT{#1}}
\newcommand{\inAppTLeftFrame}[1]{\appT{\_}{#1}}
\newcommand{\inAppTRightFrame}[1]{\appT{#1}{\_}}
\newcommand{\inFunTLeftFrame}[1]{\funT{\_}{#1}}
\newcommand{\inFunTRightFrame}[1]{\funT{#1}{\_}}
\newcommand{\inAllTFrame}[2]{\allT{#1}{#2}{\_}}
\newcommand{\inFixTLeftFrame}[1]{\fixT{\_}{#1}}
\newcommand{\inFixTRightFrame}[1]{\fixT{#1}{\_}}
\newcommand{\inLamTFrame}[2]{\lamT{#1}{#2}{\_}}

\newcommand{\inBuiltin}[5]{\builtin{#1}{#2}{#3 #4 #5}}

%% Inputs for the untyped CEK machine
\newcommand{\invalue}[1]{\texttt{\textlangle} #1 \texttt{\textrangle}}

\newcommand{\VCon}[2]{\invalue{\keyword{con} ~ #1 ~ #2}}
\newcommand{\VDelay}[2]{\invalue{\keyword{delay} ~ #1 ~ #2}}
\newcommand{\VLamAbs}[3]{\invalue{\keyword{lam} ~ #1 ~#2 ~ #3}}
\newcommand{\VBuiltin}[3]{\invalue{\keyword{builtin} ~ #1 ~ #2 ~ #3}}
 %% (builtin bn v_1 ... v_k _ m_{k+1} ... m_n, arity)

%%% CK Machine Normalization

\newcommand{\compute}{\triangleright}
\newcommand{\return}{\triangleleft}
\newcommand{\cekerror}{\mdblkdiamond}
\newcommand{\cekhalt}[1]{\mdwhtsquare\,#1}

\newcommand{\bcompute}{\color{blue}\compute}  
\newcommand{\breturn}{\color{blue}\return}  
\newcommand{\bmapsto}{\color{blue}\mapsto}
% This is to get blue symbols after an '&' inside an alignat
% environment.  It seems to mess up the spacing if you do anything
% else.



\newcommand{\inInstLeftFrame}[1]{\inst{\_}{#1}}
\newcommand{\inWrapRightFrame}[2]{\iwrap{#1}{#2}{\_}}
\newcommand{\inUnwrapFrame}{\unwrap{\_}}
\newcommand{\inAppLeftFrame}[1]{\app{\_}{#1}}
\newcommand{\inAppRightFrame}[1]{\app{#1}{\_}}

% Extra frames for untyped term normalisation
\newcommand{\inForceFrame}{\force{\_}}
\newcommand{\inBuiltinU}[4]{\builtin{#1}{#2 #3 #4}}

% These are for use inside listings and $...$.  If you just use
% \textit in listings it uses the italic tt font and the spacing
% inside the words is a bit strange.  Spacing is also bad if you
% just put something like "integer" in math text.
\newcommand\unit{\ensuremath{\mathit{unit}}}
\newcommand\one{\ensuremath{\mathit{one}}}
\renewcommand\boolean{\ensuremath{\mathit{boolean}}}
\newcommand\integer{\ensuremath{\mathit{integer}}}
\newcommand\bytestring{\ensuremath{\mathit{bytestring}}}
\newcommand\str{\ensuremath{\mathit{str}}}
\newcommand\true{\ensuremath{\mathit{true}}}
\newcommand\false{\ensuremath{\mathit{false}}}
\newcommand\case{\ensuremath{\mathit{case}}}
\newcommand\signed{\ensuremath{\mathit{signed}}}
\newcommand\txhash{\ensuremath{\mathit{txhash}}}
\newcommand\pubkey{\ensuremath{\mathit{pubkey}}}
\newcommand\blocknum{\ensuremath{\mathit{blocknum}}}
%% \newcommand\uniqmem[1]{#1^{\blacktriangledown}}  %% Unique member of size type

\newcommand\listOf[1]{\mathtt{list}\,\mathtt{(}#1\mathtt{)}}
\newcommand\pairOf[2]{\mathtt{pair}\,\mathtt{(}#1\mathtt{,}\,#2\mathtt{)}}


%% \newcommand\disj{\mathbin{\dot{\cup}}}    % Infix disjoint union
%% \newcommand\bigdisj{\bigcup^{\bullet}}      % Prefix disjoint union
\newcommand\disj{\uplus}                  % Infix disjoint union
\newcommand\bigdisj{\biguplus}            % Prefix disjoint union
\newcommand\TyOp{\mathscr{O}}             % Type operators
\newcommand\Var{\mathscr{V}}              % Type variables
\newcommand\QVar{\mathscr{Q}}             % Quantified type variables
\newcommand\Uni{\mathscr{U}}              % Universe of built-in types
\newcommand\Unihat{\hat{\mathscr{U}}}     % Universe of built-in types extended with type variables
\newcommand\UniTop{\Uni^{\top}}      % Universe of built-in types extended with a top element
\newcommand\UnihatStar{\Unihat \disj \Var_*} % Built-in type or type name
\newcommand\UnihatTop{\Unihat^{\top}}      % Universe of built-in types extended with a top element
\newcommand\op{\mathit{op}}               % Type operator names
\newcommand\valency[1]{\left| #1 \right|} % Valency of a type operator
\newcommand\Fun{\mathscr{B}}              % Built-in functions
\newcommand\Inputs{\mathscr{I}}           % Built-in functions
\newcommand\Con[1]{\mathscr{C}_{#1}}  % Constant terms of a given types
\newcommand\R{\mathscr{R}}           % Things that a builtin can return
\newcommand\bsig[1]{\lvert{#1}\rvert} % Signature |b| of a built-in function.
\newcommand\arity[1]{\alpha({#1})}   % Arity \alpha(b) of a built-in function.
\newcommand\sat[1]{\gamma({#1})} % Interleaving structure of a partial builtin application
\newcommand\fv[1]{\mathsf{FV}(#1)}
\newcommand{\N}{\mathbb{N}}
\newcommand{\Nplus}{\N^{+}}
\newcommand{\Z}{\mathbb{Z}}
\newcommand{\byte}{\mathbb{B}}
\newcommand{\tn}{\mathit{tn}}
\newcommand{\lit}[1]{\mathbf{\mathsf{lit\,_{#1}}}}
\newcommand{\barsigma}{\bar{\sigma}}

\newcommand{\kindstar}[1]{#1::\text{\textasteriskcentered}}  % "v::*"  (The ordinary * is a bit above the centreline)
\newcommand{\kindhash}[1]{#1::\text{\#}}                     % "v::#" 

\newcommand\errorX{\textbf{\texttimes}}
\newcommand\ValueOrError{\Inputs\disj\{\errorX\}}
\newcommand\withError[1]{\ensuremath{#1_{\errorX}}}



\newcommand{\cons}{\mathbin{\!::\!}}
\newcommand{\snoc}{\mathbin{\!\cdot\!}}

\newcommand{\forallty}[1]{\mathbf{\forall}#1}
\newcommand{\denote}[1]{\ensuremath{\llbracket\mathit{#1}\rrbracket}}
\newcommand{\reify}[1]{\ensuremath{\lBrace{#1}\rBrace}}
%\newcommand{\args}[1]{\textbf{\textsf{args}}(#1)}
\DeclareMathOperator{\args}{\mathbf{\mathsf{args}}}
\DeclareMathOperator{\type}{\mathbf{\mathsf{type}}}
\DeclareMathOperator{\length}{\ell}
\DeclareMathOperator{\nextArg}{\mathsf{next}}
\DeclareMathOperator{\divfn}{div}
\DeclareMathOperator{\modfn}{mod}
\DeclareMathOperator{\quotfn}{quot}
\DeclareMathOperator{\remfn}{rem}
\DeclareMathOperator{\inj}{inj}
\DeclareMathOperator{\proj}{proj}
\DeclareMathOperator{\is}{is}

%%% Spacing in tables

\newcommand\sep{4pt}
% The table of abbreviations previously had \\\\ at the end of each line, which
% made it quite long. Lines are now separated by a vertical space of size \sep.
% This makes it a bit more readable than no spacing at all, but not too long

\newcommand{\Strut}{\rule[-2mm]{0mm}{6mm}}

\newcommand{\conUnitType}{\keyword{unit}}
\newcommand{\conBooleanType}{\keyword{bool}}


\begin{document}

\maketitle

\begin{center}
  \LARGE{\red{\textsf{DRAFT}}}
\end{center}

\begin{abstract}
  This is intended to be a reference guide for developers who want to utilise
  the Plutus Core infrastructure.  We lay out the grammar and syntax of untyped
  Plutus Core terms, and their semantics and evaluation rules.  We also describe
  the built-in types and functions.  Appendix~\ref{appendix:default-builtins-alonzo}
  includes a list of supported builtins in each era and the formally verified
  behaviour.

  This document only decribes untyped Plutus Core: a subsequent version will also
  include the syntax and semantics of Typed Plutus Core and describe its relation to
  untyped Plutus Core.
\end{abstract}

\newpage
\tableofcontents
\newpage

\kwxm{This is still pretty messy.  There's a horrible mixture of syntactic
  BNF-style notation, set-theoretic notation, and type theory.  It started off
  OK but then things got a bit out of hand when I added polymorphic built-in
  functions and type operators.  I haven't made any attempt to describe what's
  going on in the implementation of the built-in machinery: instead I've written
  something that I think covers everything that we actually do with the builtins
  at the moment.  I think that the machinery is expressive enough to implement
  some quite complicated things that we don't use at the moment, but trying to
  specify all of that would involve specifying large parts of the Haskell type
  system (with numerous extensions turned on), which is not going to happen.  If
  we decided to add built-in GADTs or something then I don't think the
  presentation here would be able to handle it.  }

\kwxm{ There was a suggestion that this should be a pure specification with as
  little commentary as possible, but that's definitely not the case at the
  moment because there's a large amount of complicated notation that would be
  difficult to understand if you didn't already know what was going on.  Perhaps
  we could relegate some of the exegesis to notes at the back if we want to make
  the main text less verbose.  }

\kwxm{There's probably also quite a bit of repetition, but I think it'll need
  new readers to spot that kind of thing.}

\section{Introduction}
\label{sec:introduction}
Plutus Core is an eagerly-evaluated version of the untyped lambda calculus
extended with some ``built-in'' types and functions; it is intended for the
implementation of validation scripts on the Cardano blockchain.  This document
presents the syntax and semantics of Plutus Core, a specification of an
efficient evaluator, a description of the built-in types and functions available
in the Alonzo release of Cardano, and a specification of the binary
serialisation format used by Plutus Core.

Since Plutus Core is intended for use in an environment where
computation is potentially expensive and excessively long computations can be
problematic we have also developed a costing infrastructure for Plutus Core
programs. A description of this will be added in a later version of this
document.

We also have a typed version of Plutus Core which provides extra robustness when
untyped Plutus Core is used as a compilation target, and we will eventually
provide a specification of the type system and semantics of Typed Plutus Core
here as well, together with its relationship to untyped Plutus Core.

\section{Some Basic Notation}
\begin{itemize}
\item  The symbol $[]$ denotes an empty list.
\item The notation $[x_1, \ldots, x_n]$ denotes
  a list containing the elements $x_1, \ldots, x_n$.  If $n<1$ then the list
  is empty.
\item Given an object $x$ and a list $L = [x_1,\ldots, x_n]$,
we denote the list $[x,x_1,\ldots, x_n]$ by $x \cons L$.
\item Given a list $L = [x_1, \ldots, x_n]$ and an object $x$,
we denote the list $[x_1, \ldots, x_n, x]$ by $L \snoc x$.
%%\item We say that the list $L^{\prime}$ is a \textit{proper prefix} of the list
%%  $L = [x_1, \ldots, x_n]$, and
%%  write $L^{\prime} \prec L$,  if $L^{\prime} = [x_1, \ldots, x_m]$ for some $m<n$.
\item Given a syntactic category $V$, the symbol $\repetition{V}$ denotes a
  possibly empty list $[V_1,\ldots, V_n]$ of elements $V_i \in V$.
\end{itemize}

\section{The Grammar of Plutus Core}
\label{sec:untyped-plc-grammar}
This section presents the grammar of Plutus Core in a Lisp-like form.  This is
intended as a specification of the abstract syntax of the language; it may also
by used by tools as a concrete syntax for working with Plutus Core programs, but
this is a secondary use and we do not make any guarantees of its completeness
when used in this way.  The primary concrete form of Plutus Core programs is the
binary format described in Appendix~\ref{appendix:serialisation}.

\subsection{Lexical grammar}
\label{sec:untyped-plc}
\thispagestyle{plain}
\pagestyle{plain}

\begin{minipage}{\linewidth}
    \centering
    \[\begin{array}{lrclr}

        \textrm{Name}        & n      & ::= & \texttt{[a-zA-Z][a-zA-Z0-9\_\textquotesingle]\textsuperscript{*}}   & \textrm{name}\\

        \textrm{Var}           & x      & ::= & n & \textrm{term variable}\\
        \textrm{BuiltinName}   & bn     & ::= & n & \textrm{built-in function name}\\
        \textrm{Version} & v & ::= & \texttt{[0-9]\textsuperscript{+}.[0-9]\textsuperscript{+}.[0-9]\textsuperscript{+}}& \textrm{version}\\

        \textrm{Constant} & c & ::= & \langle{\textrm{literal constant}}\rangle& \\

    \end{array}\]
    \captionof{figure}{Lexical grammar of Plutus Core}
    \label{fig:lexical-grammar-untyped}
\end{minipage}



%   @sqs   = '  ( ($printable # ['\\])  | (\\$printable) )* '
%   
%   -- A double quoted string, allowing escaped characters including \".  Similar to @sqs
%   @dqs   = \" ( ($printable # [\"\\]) | (\\$printable) )* \"
%   
%   -- A sequence of printable characters not containing '(' or ')' such that the
%   -- first character is not a space or a single or double quote.  If there are any
%   -- further characters then they must comprise a sequence of printable characters
%   -- possibly including spaces, followed by a non-space character.  If there are
%   -- any leading or trailing spaces they will be consumed by the $white+ token
%   -- below.
%   $nonparen = $printable # [\(\)]
%   @chars = ($nonparen # ['\"$white]) ($nonparen* ($nonparen # $white))?
%   
%       <literalconst> "()" | @sqs | @dqs | @chars { tok (\p s -> alex $ TkLiteralConst p (textOf s)) `andBegin` 0 }

%% "()"
%% @sqs   = '  ( ($printable # ['\\])  | (\\$printable) )* '
%% @dqs   = \" ( ($printable # [\"\\]) | (\\$printable) )* \"
%% @chars = ($nonparen # ['\"$white]) ($nonparen* ($nonparen # $white))?


\subsection{Grammar}
\begin{minipage}{\linewidth}
    \centering
    \[\begin{array}{lrclr}
    \textrm{Term}       & L,M,N  & ::= & x                      & \textrm{variable}\\
                        &        &     & \con{tn}{c}            & \textrm{constant}\\
                        &        &     & \builtin{b}            & \textrm{builtin}\\
                        &        &     & \lamU{x}{M}            & \textrm{$\lambda$ abstraction}\\
                        &        &     & \appU{M}{N}            & \textrm{function application}\\
                        &        &     & \delay{M}              & \textrm{delay execution of a term}\\
                        &        &     & \force{M}              & \textrm{force execution of a term}\\
                        &        &     & \errorU                & \textrm{error}\\
        \textrm{Program}& P      & ::= & \version{v}{M}         & \textrm{versioned program}

    \end{array}\]
    \captionof{figure}{Grammar of untyped Plutus Core}
    \label{fig:untyped-grammar}
\end{minipage}

\paragraph{Iterated applications.}
An application of a term $M$ to a term $N$ is represented by
$\appU{M}{N}$.   We may occasionally write $\appU{M}{N_1
  \ldots N_k}$ as an abbreviation for an iterated application
$\mathtt{[}\ldots\mathtt{[[}M\;N_1\mathtt{]}\;N_2\mathtt{]}\ldots
  N_k\mathtt{]}$, and tools may also use this as concrete syntax.

\paragraph{Built-in types and functions.} The language is parameterised by a set $\Uni$ of
\textit{built-in types} (we sometimes refer to $\Uni$ as the
\textit{universe}) and a set $\Fun$ of \textit{built-in functions}
(\textit{builtins} for short), both of which are sets of Names.
Briefly, the built-in types represent sets of constants such as
integers or strings; constant expressions $\con{tn}{c}$ represent
values of the built-in types (the integer 123 or the string
\texttt{"string"}, for example), and built-in functions are functions
operating on these values, and possibly also general Plutus Core
terms.  Precise details are given in
Section~\ref{sec:specify-builtins}.  Plutus Core comes with a default
universe and a default set of builtins, which are described in
Appendix~\ref{appendix:default-builtins-alonzo}.


\section{Interpretation of built-in types and functions.}
\label{sec:specify-builtins}
As mentioned earlier, Plutus Core is generic over a universe $\Uni$ of types and
a set $\Fun$ of built-in functions.  As the terminology suggests, built-in
functions are interpreted as functions over terms and elements of the built-in
types: in this section we make this interpretation precise by giving a
specification of built-in types and functions in a set-theoretic denotational
style.  We require a considerable amount of extra notation in order to do this,
and we emphasise that nothing in this section is part of the syntax of Plutus
Core: it is meta-notation introduced purely for specification purposes.


\paragraph{Set-theoretic notation.}
We begin with some extra set-theoretic notation:
\begin{itemize}
  \item $\N = \{0,1,2,3,\ldots\}$.
  \item $\Nplus = \{1,2,3,\ldots\}$.
  \item $\Z = \{\ldots, -2, -1, 0, 1, 2, \ldots\}$
  \item $\byte = \{n \in \Z: 0 \leq n \leq 255\}$, the set of 8-bit bytes.
\item The symbol $\disj$ denotes a disjoint union of sets;  for emphasis we often use this
  to denote the union of sets which we know to be disjoint.
\item Given a set $X$, $X^*$ denotes the set of finite sequences of elements of
  $X$:
$$
  X^*= \bigdisj{\{X^n: n \in \mathbb{N}\}}.
  $$
  \kwxm{This is conflated with the list notation introduced earlier, but I don't think we say that anywhere.}
\item We assume that there is a special symbol $\errorX$ which does not appear
  in any other set we mention.  The symbol $\errorX$ is used to indicate that
  some sort of error condition has occurred, and we will often need to consider
  situations in which a value is either $\errorX$ or a member of some set $S$.
  For brevity, if $S$ is a set then we define
      $$
      \withError{S} := S \disj \{\errorX\}.
      $$
\end{itemize}

\subsection{Built-in types}
\label{sec:built-in-types}
We require some extra syntactic notation for built-in types: see Figure~\ref{fig:type-names-operators}.

\begin{minipage}{\linewidth}
    \centering
    \[\begin{array}{rclr}
    \mathit{atn}    & ::= & n & \textrm{Atomic type}\\
     op             & ::= & n & \textrm{Type operator}\\
     tn             & ::= & \mathit{atn} \ | \ op(tn,tn,...,tn) & \textrm{Type}\\
    \end{array}\]
    \captionof{figure}{Type names and operators}
    \label{fig:type-names-operators}
\end{minipage}

\medskip
\noindent
We assume that we have a set $\Uni_0$ of \textit{atomic type names} and a set
$\TyOp$ of \textit{type operator names}.  Each type operator name $\op \in
\TyOp$ has an \textit{argument count} $\valency{\op} \in \Nplus$, and a type name
$\op(tn_1, \ldots, \tn_n)$ is well-formed if and only if $n = \valency{\op}$.
We define the \textit{universe} $\Uni$ to be the closure of $\Uni_0$ under repeated
applications of operators in $\TyOp$:
$$
\Uni_{i+1} = \Uni_i \cup \{\op(tn_1, \ldots, tn_{\valency{\op}}): \op \in \TyOp, tn_1, \ldots, tn_{\valency{op}} \in \Uni_i\}
$$
$$
\Uni = \bigcup\{\Uni_i: i \in \Nplus\}
$$

\kwxm{Maybe we could have a judgment like $\Uni \vdash t\ \textsf{type}$
  and use inference rules instead of sets.  That would amount to the same thing but
would be considerably less compact.}

\kwxm{I'm inconsistently using ``type'' and ``type name'' for the things in
  $\Uni$, and that's further complicated by the introduction of polymorphic types later.}

The universe $\Uni$ consists entirely of \textit{names}, and the semantics of
these names are given by \textit{denotations}. Each type name $\tn \in \Uni$ is
associated with some mathematical set $\denote{tn}$, the \textit{denotation} of
$\tn$. For example, we might have $\denote{\texttt{boolean}}= \{\mathsf{true},
\mathsf{false}\}$ and $\denote{\texttt{integer}} = \mathbb{Z}$ and
$\denote{\pairOf{a}{b}} = \denote{a} \times \denote{b}$.  See
Appendix~\ref{appendix:default-builtins-alonzo} for a description of the built-in
types and type operators available in The Alonzo release of Plutus Core.

For non-atomic type names $tn = \op(tn_1, \ldots, \tn_r)$ we require the
denotation of $tn$ to be obtained in some uniform way from the denotations of
$tn_1, \ldots, \tn_r$.  
\kwxm{The thing about denotations of non-atomic type names being obtained in
  ``in some uniform way'' from the argument types is more than a little bit
  vague.}

\subsubsection{Type Variables}
Built-in functions can be polymorphic, and to deal with this we need
\textit{type variables}.  An argument of a polymorphic function can be either
restricted to built-in types or can be an arbitrary term, and we define two
different kinds of type variables to cover these two situations.  See
Figure~\ref{fig:type-variables}, where we also define a class of
\textit{quantifications} which are used to introduce type variables.

\begin{minipage}{\linewidth}
  \centering
      \[\begin{array}{lrclr}
        \textrm{TypeVariable}    & tv      & ::= & n_{*} & \textrm{fully polymorphic type variable}\\
                                 &         &     & n_{\#} & \textrm{built-in-polymorphic type variable}\\
        \textrm{Quantification}  & q       & ::= & \forallty{tv} & \textrm{quantification}\\
        
    \end{array}\]
    \captionof{figure}{Type variables}
    \label{fig:type-variables}
\end{minipage}

\medskip
\noindent
We denote the set of all possbile quantifications by $\QVar$, the set of all
possible type variables by $\Var$, the set of all fully-polymorphic type
variables by $\Var_*$, and the set of all built-in-polymorphic type variables
$v_\#$by $\Var_\#$.  Note that $\Var \cup \Uni = \varnothing$ since the symbols
${}_*$ and ${}_\#$ do not occur in $\Uni$.

The two kinds of type variable are required because we have two different types
of polymorphism. Later on we will see that built-in functions can take arguments
which can be of a type which is unknown but must be in $\Uni$, whereas other
arguments can range over a larger set of values such as the set of all Plutus
Core terms. Type variables in $\Var_\#$ are used in the former situation
and $\Var_*$ in the latter.

Given a variable $v \in \Var$ we sometimes write
$$
   \kindhash{v} \mbox{\quad if $v \in \Var_\#$}
$$
and
$$
   \kindstar{v} \mbox{\quad if $v \in \Var_*$}
$$

\kwxm{I'm using syntax to represent kinds here.  I haven't introduced actual
  kinds because (a) we don't have a proper type system in Plutus Core (yet), and
  (b) the \texttt{TypeScheme} type in the implementation only has one kind of
  type variable.  I'm only using this in the specification of the signatures of
  built-in functions to characterise some aspects how the builtins machinery
  works in practice: things in $\Var_*\backslash\Var_\#$ can fail with an
  unlifting error at runtime but that's not statically enforced anywhere.}
\kwxm{The $::$ notation risks confusion with the list cons notation.}
We also need to talk about polymorphic types, and to do this we define an
extended universe of types $\Unihat$ by adjoining $\Var_\#$ to $\Uni_0$ and
closing under type operators as before:

$$
\Unihat_0 = \Uni_0 \cup \Var_\#
$$
$$
\Unihat_{i+1} = \Unihat_i \cup \{\op(tn_1, \ldots, tn_{\valency{\op}}): \op \in \TyOp, tn_1, \ldots, tn_{\valency{op}} \in \Unihat_i\}
$$
$$
\Unihat = \bigcup\{\Unihat_i: i \in \Nplus\}
$$

\noindent We define the set of \textit{free variables} of an element of $\Unihat$ by
$$
\fv{tn} = \varnothing \ \mbox{if $tn \in \Uni_0$}
$$
$$
\fv{v_\#} = \{v_\#\}
$$
$$
\fv{\op(tn_1, \ldots, tn_k)} = \fv{tn_1} \cup \fv{tn_2} \cup \cdots \cup \fv{tn_r}
$$
  
\noindent Thus $\fv{tn} \subseteq \Var_\#$ for all $tn \in \Uni$.
We say that a type name $tn \in \Unihat$ is \textit{monomorphic} if $\fv{tn} =
\varnothing$; otherwise $tn$ is \textit{polymorphic}.  The fact that type variables
in $\Unihat$ are only allowed to come from $\Var_\#$ will ensure that values of
polymorphic types such as lists and pairs can only contain values of built-in
types: in particular, we will not be able to construct types representing things
such as lists of Plutus Core terms.

\kwxm{For type operators, ``polymorphic'' really means ``polymorphic over
  built-in types''.}

\kwxm{$\Uni$ is the set of built-in types and $\Unihat$ is
  that set extended to include polymorphic types as well.  Later on we quite
  often have to look at $\Unihat\backslash \Uni$ to talk about types that really
  are polymorphic. Maybe it would be better to have a separate universe of
  polymorphic types called $\Uni_{\#}$ or something, but then we'd also have to
  talk about $\Uni \disj \Uni_{\#}$ instead.  Maybe we could define $\Unihat =
  \Uni \disj \Uni_{\#}$?
  }

\subsection{Arguments of built-in functions}
\label{sec:builtin-inputs}
To treat the typed and untyped versions of Plutus Core uniformly it is necessary
to make the machinery of built-in functions generic over a set $\Inputs$ of
\textit{inputs} which are taken as arguments by built-in functions.  In practice
$\Inputs$ will be the set of Plutus Core values or something very closely
related.

\indent We require $\Inputs$ to have the following properties:
\begin{itemize}
\item $\Inputs$ is disjoint from $\denote{\tn}$ for all $\tn \in \Uni$
\item We require disjoint subsets $\Con{\tn} \subseteq \Inputs$ ($\tn \in \Uni$)
  of \textit{constants of type $\tn$} and maps $\denote{\cdot}_{\tn}: \Con{\tn}
  \rightarrow \denote{\tn}$ (\textit{denotation}) and $\reify{\cdot}_{\tn}:
  \denote{\tn} \rightarrow \Con{\tn}$ (\textit{reification}) such that
  $\reify{\denote{c}_{\tn}}_{\tn} = c \mbox{ for all } c \in \Con{\tn}$.  We do
  not require these maps to be bijective (for example, there may be multiple
  inputs with the same denotation), but the condition implies that
  $\denote{\cdot}_{\tn}$ is surjective and $\reify{\cdot}_{\tn}$ is injective.
\end{itemize}

\kwxm{I'm still not sure what to call the things what we feed to builtins.
  Previously there were called ``terms'', but in our setting they're \textit{not}
  actually arbitrary terms.  I tried ``values'' instead, but that was confusing.
  ``Inputs'' is a bit better, but (a) they're also what builtins
  \textit{output}, and (b) there's a small risk of confusion with UTXO inputs.}

\noindent For example, we could take $\Inputs$ to be the set of all Plutus Core
values (see Section~\ref{sec:uplc-values}), $\Con{\tn}$ to be the set of all
terms $\con{tn}{c}$, and $\denote{\cdot}_{\tn}$ to be the function which maps
$\con{tn}{c}$ to $c$.  For simplicity we are assuming that mathematical entities
occurring as members of type denotations $\denote{\tn}$ are embedded directly as
values $c$ in Plutus Core constant terms. In reality, tools which work with
Plutus Core will need some concrete syntactic representation of constants;
we do not specify this here, but see Section~\ref{sec:alonzo-built-in-types} for
suggested syntax for the built-in types included in the Alonzo release.

We will consistently use the symbol $\tau$ (and subscripted versions of it)
to denote a member of $\UnihatStar$ in the rest of the document.
\kwxm{Check that we really are doing that consistently.}


\subsection{Built-in functions}
\label{sec:builtin-functions}

\paragraph{Signatures.}
Every built-in function $b \in \Fun$ has a \textit{signature} $\sigma(b)$ of the form
$$[\iota_1, \ldots, \iota_n] \rightarrow \tau$$
with
\begin{itemize}
  \item $\iota_j \in \UnihatStar \disj \QVar \enspace\mbox{for all $j$}$
  \item $\tau \in \UnihatStar$
  \item $\lvert\{j : \iota_j \notin \QVar\}\rvert \geq 1$ (so $n \geq 1$)
  \item If $\iota_j$ involves $v \in \Var$ then $\iota_k = \forallty{v}$ for
    some $k < j$, and similarly for $\tau$; in other words, any type variable
    $v$ must be introduced by a quantification before it is used. (Here $\iota$
    \textit{involves} $v$ if either $\iota = tn \in \Unihat$ and $v \in \fv{tn}$
    or $\iota = v$ and $\kindstar{v}$.)
  \item If $j \neq k$ and $\iota_j, \iota_k \in \QVar$ then $\iota_j \neq
    \iota_k$; ie, no quantification appears more than once.
\end{itemize}
\noindent 


\noindent For example, in our default set of built-in functions we have the
functions \texttt{mkCons} with signature $[\forall a_\#, a_\#,
  \listOf{a_\#}] \rightarrow \listOf{a_\#}$ and \texttt{ifThenElse} with signature
$[\forall a_*, \mathtt{boolean}, a_*, a_*] \rightarrow a_*$.  When we use
\texttt{mkCons} its arguments must be of built-in types, but the two final
arguments of \texttt{ifThenElse} can be any Plutus Core values.

\kwxm{Can we write $\mathtt{mkPair}: [\forall a_\#, a_\#, \forall b_\#, b_\#] \rightarrow pair(a_\#,b_\#)$?}

\noindent If $b$ has signature $[\iota_1, \ldots, \iota_n] \rightarrow \tau$ then the \textit{arity}
of $b$ is
$$
  \alpha(b) = [\iota_1, \ldots, \iota_n]
$$
and the \textit{argument count} of $b$ is $\left|b\right| = n$
  

\noindent We may abuse notation slightly by using the symbol $\sigma$ to denote
a specific signature as well as the function which maps built-in function names
to signatures, and similarly with the symbol $\alpha$.

\medskip
\noindent Given a signature
$\sigma = [\iota_1, \ldots, \iota_n] \rightarrow \tau$,
we define the \textit{reduced signature} $\barsigma$ to be
$$
\barsigma = [\iota_j : \iota_j \notin \QVar] \rightarrow \tau
$$

We extend the usual set comprehension notation to lists in the obvious way, so
this just denotes the signature $\sigma$ with all quantifications omitted. We
will often write a reduced signature in the form $[\tau_1, \ldots, \tau_m]
\rightarrow \tau$ to emphasise that the entries are \textit{types}, and
$\mathbf{\forall}$ does not appear.

What is the intended meaning of this notation?  In Typed Plutus Core we have to
instantiate polymorphic functions (both built-in functions and polymorphic
lambda terms) at concrete types before they can be applied, and in Untyped
Plutus Core instantiation is replaced by an application of \texttt{force}.  When
we are applying a built-in function we supply its arguments one by one, and we
can also apply \texttt{force} (or perform type instantiation in the typed case)
to a partially-applied builtin ``between'' arguments (and also after the final
argument); no computation occurs until all arguments have been supplied and all
\texttt{force}s have been applied. The signature (read from left to right)
specifies what types of arguments are expected and how they should be
interleaved with applications of \texttt{force}. A fully-polymorphic type
variable $a_*$ indicates that an arbitrary value from $\Inputs$ can be provided,
whereas a type from $\Unihat$ indicates that a value of the specified built-in
type is expected. Occurrences of quantifications indicate that \texttt{force} is
to be applied to a partially-applied builtin; we allow this purely so that
partially-applied builtins can be treated in the same way as delayed
lambda-abstractions: \texttt{force} has no effect unless it is the very last
item in the signature).  In Plutus Core, partially-applied builtins are values
which can be treated like any others (for example, by being passed as an
argument to a \texttt{lam}-expression): see Section~\ref{sec:uplc-values}.

To make some of the above remarks more precise and simplify some of the later
exposition we introduce a relation $\sim \enspace \subseteq \Inputs \times
(\UnihatStar \disj \QVar)$ of \textit{compatibility} between inputs and
signature entries: this is defined in Figure~\ref{fig:compatibility}.

\begin{figure}[H]
  \centering
  $$
  \begin{array}{lll}
  V \sim \iota \enspace & \mbox{if} & \iota \in \Uni \mbox{ and $V \in \Con{\iota}$}\\
     & \mbox{or} & \iota \in \Unihat \backslash \Uni \\ %%\mbox{and $V \in \Con_{tn}$ for some $\tn \in \Uni$}\\
     & \mbox{or} & \iota \in \Var_*
  \end{array}
  $$
  \caption{Compatibility of inputs with signature entries}
  \label{fig:compatibility}
\end{figure}

\kwxm{This bit of notation really helps: I previously had to repeat the three
  cases in a number of places (in the CEK machine, for example).}

\noindent Note that we can never have $V \sim \forall v$.

\paragraph{Denotations of built-in functions.}
If we have a built-in function $b$ with reduced signature
$$
      \barsigma(b) = [\tau_1, \ldots, \tau_m] \rightarrow \tau,
$$

then we require $b$ to have a \textit{denotation} (or \textit{meaning}), a function
      
$$
\denote{b}: \denote{\tau_1} \times \cdots \times \denote{\tau_m} \rightarrow \withError{\denote{\tau}}
$$

\noindent where for a name $a$
$$
\denote{a_\#} = \bigdisj\{\denote{tn}: tn \in \Uni\}
$$
and
$$
\denote{a_*} = \Inputs.
$$


Denotations of builtins are mathematical functions which terminate on every
possible input; the symbol $\errorX$ can be returned by a function to indicate
that something has gone wrong, such as an attempted division by zero.

\medskip\noindent
If $r$ is the result of the evaluation of some built-in function there are thus
three possibilities:
\begin{enumerate}
\item $r \in \denote{\tn} \enspace \mbox{for some $\tn \in \Uni$}$
\item $r \in \Inputs$
\item $r = \errorX$
\end{enumerate}
In other words,
$$
r \in \R := \bigdisj\{\denote{\tn}: \tn \in \Uni \} \disj \Inputs \disj \{\errorX\}.
$$

\noindent Our assumptions on the set $\Inputs$
(Section~\ref{sec:builtin-inputs}) allow us define a function
$$
\reify{-}: \R \rightarrow \withError{\Inputs}
$$
which converts results of built-in functions back into inputs (or the $\errorX$ symbol)
\begin{enumerate}
\item If $r \in \denote{tn}$, then we let $\reify{r} = \reify{r}_{\tn} \in \Con{\tn} \subseteq \Inputs$.
\item If $r \in \Inputs$ then we let $\reify{r} = r$
\item We let $\reify{\errorX} = \errorX$
\end{enumerate}
\kwxm{We have to use $\Inputs \disj \{\errorX\}$ to deal with the fact that
  $\errorU$ isn't a value, so we have to defer handling errors until later.}
\kwxm{The notation here is maybe a bit confusing.  The $\Con{\tn}$ live in the
  syntactic world (where they're subsets of $\Inputs$) and the $\denote{tn}$
  live in the world of sets; however $\Inputs$ lives in \textit{both} worlds,
  and it's disjoint from all of the $\denote{tn}$ in the world of sets but not
  in the world of syntax.  I think we do need something like this because
  sometimes a builtin argument must be a constant but at other times it can be
  an arbitrary value, which includes all of the constants.}

\paragraph{Behaviour of built-in functions.}  
A built-in function $b$ can only inspect arguments which are values of built-in
types; other arguments (occurring as $a_*$ in $\barsigma(b)$) are treated opaquely,
and can be discarded or returned as (part of) a result, but cannot be altered or
examined (in particular, they cannot be compared for equality): $b$ is
\textit{parametrically polymorphic} in such arguments.  This implies that if a
builtin returns a value $v \in \Inputs$, then $v$ must have been an argument of
the function.

\roman{This is not quite true, unfortunately. \texttt{toBuiltinMeaning}
  constrains \texttt{term} with \texttt{HasConstantIn uni term}. This means that we can
  check whether an argument whose type is a type variable is a constant or
  not. And if it's a constant, we can obtain the type tag and do all kinds of
  fancy things with it. For example a builtin checking if two values of
  different types are equal constants is representable. This breaks parametricity.
}

\kwxm{I think we should just require that people don't do that. $\uparrow$}

\kwxm{I think we can express the parametricity property as follows.  Suppose $b$
  is a builtin with constant arguments $x_1,\ldots, x_m$ and ``term'' arguments
  $y_1, \ldots, y_n$.  Then there are two cases: (a) $b$ returns a constant, and
  (b) $b$ returns a term (you can tell which of these is true by looking at
  $\sigma(b)$).  In case (a), the result of applying $b$ should depend only on
  the values of the $x_i$ and be completely independent of the values of the
  $y_j$, and even of what $\Inputs$ is.  In case (b), for a given choice of the
  $x_i$, the output must be some $y_k$, and $k$ should be the same for all
  choices of $y_1, \ldots, y_n$, and again it should even be independent of the
  choice of $\Inputs$.}

We also require built-in functions to be parametrically polymorphic in arguments
which are of polymorphic built-in types, such as lists, and when a function
signature contains type variables in $\Var_\#$ we will expect the actual
arguments supplied during application to have consistent types (for a given type
variable $a_\#$, all arguments to which it refers should have the same built-in
type at run time).  However we do not enforce this in the notation above:
instead consistency conditions of this sort will be included in the
specifications of the semantics of the full Plutus Core language.

\kwxm{The thing about relegating consistent instantiation of
  built-in-polymorphic type variables to the full semantics is there because (a)
  it's non-trivial to specify, and (b) in fact it's not enforced by the builtin
  application machinery, but must be checked dynamically in the implementation
  of the builtin when it's fully applied (\texttt{mkCons} is an example of
  this). It would be perfectly possible (I think) to implement a builtin with
  signature $[\forall a_\#, a_\#, a_\#] \rightarrow a_\#$ which could be
  successfully applied to arguments of two different built-in types, returning a
  constant of a third type.  Let's assume that builtins always check this
  stuff.}
  
When (the meaning of) a built-in function $b$ is applied (perhaps partially) to
arguments, the types of constant arguments must correspond to the types in
$\barsigma(b)$, and the function will return $\errorX$ if this is not the
case; builtins may also return $\errorX$ in other circumstances, for example if
an argument is out of range.

\kwxm{A lot of the complexity we have here is due to the fact that we've got
  explicit \texttt{delay} and \texttt{force} instead of the usual $\lambda().M$
  and $M()$.  We use the explicit version because experiments showed that it was
  noticeably faster (and we have a lot of these due to erasure of type-level
  abstraction/instantiation).  Also, who's to say that the universe contains a
  unit type?
}

%% \begin{minipage}{\linewidth}
%%     \centering \[\begin{array}{llll}

%%     t & ::= & b   &\mbox{for } b \in \Uni  \\
%%     & & \typearg                 \\
%%     & & \star               \\
%%     & & t \rightarrow t          \\
%%     \end{array}\]
%% \end{minipage}

  


\section{Term Reduction}
\label{sec:reduction}

\kwxm{
  What can happen when we have \texttt{[((builtin $b$ ...) $V$]} (maybe
interleaved with some forces) with $\iota$ the head of the remaining $\alpha$
(ie $\iota$ tells you the what the next thing $b$'s expecting is)?


Since $\iota \in \UnihatStar \disj \QVar$ there are four possibilities:
\begin{itemize}
\item $\iota \in \Unihat$.  Then there are two possibilities:
  \begin{itemize}
    \item $\iota \in \Uni$.  Then $\iota$ is a monomorphic built-in type and we expect $V$ to
    have exactly that type.  So $V$ must lie in $\Con{\iota}$, and we fail immediately if it doesn't
  \item $\iota \in \Unihat \backslash \Uni$.  Then $\iota$ is a polymorphic built-in type, so
    we check that $V$ lies in \textit{some} $\Con{tn}$: if it does then we continue, deferring
    full type checks until the application becomes saturated; if it doesn't, we fail.
  \end{itemize}
\item $\iota \in \Var_*$.  Any term is acceptable, no matter what its type, so we carry on regardless.
\item $\iota \in \QVar$. In this case a \texttt{force} is expected, so we fail.
\end{itemize}
The semantics and the evaluator both have to deal with all of these cases, which makes things
quite complex.
}


This section defines the semantics of (untyped) Plutus Core.

\subsection{Values in Plutus Core}
\label{sec:uplc-values}
The semantics of
built-in functions in Plutus Core are obtained by instantiating the sets
$\Con{tn}$ of constants of type $tn$ (see Section~\ref{sec:builtin-inputs})
to be the expressions of the form \texttt{(con} $tn$ $c$\texttt{)} and the set
$\Inputs$ to be the set of Plutus Core
\textit{values}, terms which cannot immediately undergo any further reduction,
such as lambda terms and delayed terms.
Values also include partial applications of built-in functions such as
\texttt{[(builtin modInteger) (con integer 5)]}, which cannot perform any
computation until a second integer argument is supplied.  However, partial
applications must also be \textit{well-formed}: for example, applications of
\texttt{force} must be correctly interleaved with genuine arguments, and the
arguments must (a) themselves be values, and (2) must be of the types expected
by the function, so if \texttt{modInteger} has signature $\mathtt{[integer,
    integer]} \rightarrow \mathtt{integer}$ then \texttt{[(builtin modInteger)
    (con string "green")]} is illegal.
%
The occurrence of partially-applied builtins complicates the definition of
general values considerably.

We define syntactic classes $V$ of Plutus Core values and $P$ of partial builtin
applications simultaneously:

\begin{minipage}{\linewidth}
    \centering
    \[\begin{array}{lrcl}
        \textrm{Value}  & V   & ::= & \con{tn}{c} \\
                        &     &     & \delay{M} \\
                        &     &     & \lamU{x}{M} \\
                        &     &     & A
    \end{array}\]
    \captionof{figure}{Values in Plutus Core}
    \label{fig:untyped-cek-values}
\end{minipage}

\medskip
\noindent Here $A$ is the class of well-formed partial applications, and to define
this we first define a class of possibly ill-formed iterated applications for
each built-in function $b \in \Fun$:

\begin{minipage}{\linewidth}
    \centering
  \[\begin{array}{lrl}
  P & ::= & \builtin{b}\\
    &     & \appU{P}{V}\\
    &    & \force{P}\\
    \end{array}\]
    \captionof{figure}{Partial built-in function application}
    \label{fig:partial-applications}
\end{minipage}

\medskip
\noindent We let $\textsf{P}$ denote the set of terms generated by the grammar
in Figure~\ref{fig:partial-applications} and 
we define a function $\beta$ which extracts the name of the built-in
function occurring in a term in $\textsf{P}$:
$$
 \begin{array}{ll}
 \beta(\builtin{b}) &= b\\
 \beta(\appU{P}{V}) & =\beta(P)\\
 \beta({\force{P}}) & =\beta(P)\\
\end{array}
$$


%% $$
%% \begin{array}{ll}
%%   \sat{\builtin{b}} &= []\\
%%   \sat{\appU{P}{V}} &= \sat{P}\snoc\type(V)\\
%%   \sat{\force{P}}   &= \sat{P}\snoc\fforce\\
%% \end{array}
%% $$

\noindent We also define a function $\length$ which measures the size of a term $P \in \textsf{P}$:
$$
\begin{array}{ll}
\length(\texttt{(builtin $b$)}) &= 0\\
\length(\texttt{[$P$ $V$]}) &= 1+\length(P)\\
\length(\texttt{(force $P$)}) & = 1+\length(P)
\end{array}
$$


%% \item Our built-in functions can take general members of $\Inputs$ as arguments
%%   as well as elements of the sets $\denote{\tn}$ and the symbol $\top$ is used
%%   to denote the type of elements of $\Inputs$. We use the symbol $\top$
%%   (which we assume does not appear in any other set we mention) to denote the
%%   type of non-constant elements of $\Inputs$ and write $\UniTop = \Uni \disj
%%   \{\top\}$ and $\UnihatTop = \Unihat \disj \{\top\}$.
%% \item We should be able to examine inputs (even during execution) to determine
%%   their types.  More precisely we assume that there is a function $\type:
%%   \Inputs \rightarrow \UniTop$ such that
%%   $$\type(x) =
%%   \begin{cases}
%%     \tn &\ \mbox{if } x \in \Con{tn} \mbox{ for some } tn \in \Uni\\
%%     \top &\mbox{otherwise}
%%   \end{cases}
%%   $$
%%   \noindent This is well defined because of our assumption that the sets $\Con{tn}$ are disjoint.
%% \item We also define a partial order $\preceq$  on the set $\Uni^{\top}$ by
%%   $t_1 \preceq t_2$ if $t_1 = t_2$ or $t_2 = \top$.   


\paragraph{Well-formed iterated applications.} A term $P \in \textsf{P}$ is
an application of $b = \beta(P)$ to a number of values in $S$, interleaved with
applications of $\texttt{force}$.  We now define what it means for $P$ to be
\textit{well-formed}.  Firstly we let $n = \left|b\right| $ and we require that
$l \leq n$, so that $b$ is not over-applied.  In this case we put
$\iota=\iota_l$, the element of $b$'s signature which describes what kind of
``argument'' $b$ currently expects.  We complete the definition by induction on
the structure of $P$:
\begin{enumerate}
\item $P=\mathtt{(builtin}\ b \mathtt{)}$ is always well-formed.
\item $P=\mathtt{(force}\ P^{\prime}\mathtt{)}$ is well-formed if $P^{\prime}$ is well-formed and $\iota \in \QVar$.
\item $P=\mathtt{[}P^{\prime}\ V\mathtt{]}$ is well-formed if $P^{\prime}$ is
  well-formed and $V \sim \iota$ (see Figure~\ref{fig:compatibility} for the
  definition of $\sim$).
\item Furthermore, if $l = n$ then we require that built-in polymorphic types
  are used consistently in $P$.
\end{enumerate}

\kwxm{Note that apart from type names all of this stuff is meta-notation that is
  need to describe the builtins machinery but isn't part of the language.}

\noindent Conditions (2) and (3) say the arguments of $b$ are properly
interleaved with occurrences of \texttt{force}, and that the arguments are of
the expected types.  For type consistency, the compatibility condition says that
(a) if the signature specifies a monomorphic built-in type then the type of $V$
must match it exactly; (b) if the signature specifies a polymorphic built-in
type then $V$ must be a constant of \textit{some} built-in type; and (c) if the
signature specifies a full-polymorphic type then any input is acceptable.

In case (b) further checks will be required if and when $b$ becomes fully
applied, to make sure that polymorphic type variables are instantiated
consistently.

\paragraph{Consistency of arguments and signatures.}
The meaning of condition (4) should be fairly obvious; for example if we have a
builtin $b$ with signature $[\forall a_\#, \forall b_\#, a_\#, \listOf{a_\#},
  \pairOf{a_\#}{b_\#}] \rightarrow \pairOf{\listOf{a_\#}}{\listOf{b_\#}}$ then
in a well-formed saturated application \texttt{[(builtin $b$) $U$ $V$ $W$]}
there must be (monomorphic) types $t, u \in \Uni$ such that $U$ is a constant of
type $t$, $V$ is a constant of type $\listOf{t}$, and $W$ is a constant of type
$\pairOf{t}{u}$.  A full definition of consistency will be added in a subsequent
version of this document.  We will define consistency to be a binary relation
$\approx$ between lists of values and reduced arities and we will use this
notation later in the document even though the full definition is not available yet.
\kwxm{Presumably this is some standard unification thing, but I haven't pinned that down yet.}
\kwxm{In fact the machinery is a bit laxer than this.  If we have
  \texttt{[(builtin modInteger) (con string "x")]} then that will fail
  immediately (in the evaluator), but for polymorphic builtins the consistency
  check doesn't happen until we get into the definition of the builtin, and it's
  up to the implementation to do the check (cf \texttt{mkCons}).  It's
  conceivable that a builtin might \textit{not} check for consistency: for
  instance I think we could have a builtin with signature
  $[\forall a_\#, a_\#, a_\#] \rightarrow a_\#$
  which would just throw away its second argument and
  return the first without ever checking the types.  Note also that no
  consistency checking can happen (with the current implementation) until we
  have all the arguments.  If we had a builtin with signature
  $[\forall a_\#, a_\#, a_\#, \texttt{integer}] \rightarrow a_\#$
  (we don't have anything like this at the moment) then we could apply it to a
  boolean and a string and get a partial application which according to the
  current definition (and implementation) would be well-formed.}

  \kwxm{Maybe we could have a judgment $\mathsf{wf}$ and define it using inference rules?}

\noindent We can now complete the definition of values in Figure~\ref{fig:untyped-cek-values}
by defining $A$ to be the set of well-formed \textit{partial} built-in function applications
$$
A = \{P \in \textsf{P}: P \mbox{ is well-formed and } \length(P) < \left|\alpha(\beta(P))\right|\}
$$




\paragraph{More notation.} Suppose that $A$ is a well-formed partial application with
$\beta(A) = b$, $\alpha(b) = [\iota_1,\ldots,\iota_n]$, and $\length(A) = l$.
We define a function $\nextArg$ which extracts the next argument (or
\texttt{force}) expected by $A$:
$$
    \nextArg(A) = \iota_{l+1}
$$
\noindent
This makes sense because in a well-formed partial application we have $l<n$.

\medskip
\noindent We also define a function $\args{}$ which extracts the arguments which
$b$ has received so far in $A$:
$$
\begin{array}{ll}
  \args(\builtin{b}) &= []\\
  \args(\appU{A}{V}) &= (\args(A))\snoc V\\
  \args(\force{A})   &= \args(A)\\
\end{array}
$$

\subsection{Term reduction}

%% ---------------- Grammar of Reduction Frames ---------------- %%
\kwxm{Explain what this stuff means. Remember that when we apply the reduction
  rules we always use the first applicable one.

  I'm somewhat tempted to dump this in favour of SOS.}
\kwxm{Do we need a uniqueness condition on names somewhere?}
    
\kwxm{I'm not entirely confident about these rules, especially the third rule
  (for unsaturated builtin application).  This is rather confusing because the
  first \texttt{[$A$ $V$]} is a frame \texttt{[$A$ \_]} containing a $V$ and the
  second one is a value.  The value is guaranteed to be well formed because $A$
  is by definition (maybe we should call these $W$ or something to make it
  clearer that they only represent \textit{well-formed} partial applications)
  and the premises ensure that \texttt{[$A$ $V$]} is too.  I'm still not
  convinced though.  If we have \texttt{[(builtin addInteger) (con string "x")]}
  then that's illegal but I think we won't see that because we're only supposed
  to look at the frame \texttt{[$V$ \_]} when the hole is filled with a redex,
  and \texttt{(con string "x")} isn't a redex.}

We define the semantics of Plutus Core using contextual semantics (or reduction
semantics): see~\cite{Felleisen-Hieb} or~\cite[5.3]{Harper:PFPL}, for example.
We use $A$ to denote a partial application of a built-in function as in
Section~\ref{sec:uplc-values} above.  For builtin evaluation, we
instantiate the set $\Inputs$ of Section~\ref{sec:builtin-inputs} to be the set
of Plutus Core values.  Thus all builtins take values as arguments and return a
value or $\errorX$.  Since values are terms here, we can take $\reify{V} = V$.
\kwxm{Eh?}

\newcommand\Eval[2]{\mathsf{Eval}\,(#1,#2)}

\begin{figure}[H]
\begin{subfigure}[c]{\linewidth}
    \centering
    \[\begin{array}{lrclr}
        \textrm{Frame} & f  & ::=   & \inAppLeftFrame{M}          & \textrm{left application}\\
                       &   &     & \inAppRightFrame{V}            & \textrm{right application}\\
                       &   &     & \inForceFrame                  & \textrm{force}
    \end{array}\]
    \caption{Grammar of reduction frames for Plutus Core}
    \label{fig:untyped-reduction-frames}
\end{subfigure}
%\end{figure}

\bigskip
%\begin{figure}[H]
%\ContinuedFloat
%% ---------------- Reduction via Contextual Semantics ---------------- %%
\begin{subfigure}[c]{\linewidth}
  % \def\labelSpacing{20pt}

  \judgmentdef{$\step{M}{M'}$}{Term $M$ reduces in one step to term $M'$.}

   % [(lam x M) V] -> [V/x]M
    \begin{prooftree}
        \AxiomC{}
        % \RightLabel{\textsf{apply-lambda}}
        \UnaryInfC{$\step{\app{\lamU{x}{M}}{V}}{\subst{V}{x}{M}}$}
    \end{prooftree}

    % [A V] saturated
    \begin{prooftree}
      \AxiomC{$\length(A) = \left|\beta(\alpha(A))\right|-1$}
      \AxiomC{$V \sim \nextArg(A)$}
        % \RightLabel{\textsf{final-apply}}
        \BinaryInfC{$\step{\app{A}{V}}{\Eval{\beta(A)}{(\args(A))\snoc V}}$}
    \end{prooftree}

    % [A V] unsaturated
    \begin{prooftree}
      \AxiomC{$\length(A) < \left|\beta(\alpha(A))\right|-1$}
      \AxiomC{$V \sim \nextArg(A)$}
        % \RightLabel{\textsf{intermediate-apply}}
        \BinaryInfC{$\step{\app{A}{V}}{\app{A}{V}}$}
    \end{prooftree}

    % force (delay M) -> M
    \begin{prooftree}
        \AxiomC{}
        % \RightLabel{\textsf{force-delay}}
        \UnaryInfC{$\step{\force{\delay{M}}}{M}$}
    \end{prooftree}

    % Saturated force
    \begin{prooftree}
      \AxiomC{$\length(A) = \left|\beta(\alpha(A))\right|-1$}
      \AxiomC{$\nextArg(A) \in \QVar$}
        % \RightLabel{\textsf{final-force}}
        \BinaryInfC{$\step{\force{A}}{\Eval{\beta(A)}{\args(A)}}$}
    \end{prooftree}


    % Unsaturated force
    \begin{prooftree}
      \AxiomC{$\length(A) < \left|\beta(\alpha(A))\right|-1$}
      \AxiomC{$\nextArg(A) \in \QVar$}
        % \RightLabel{\textsf{intermediate-force}}
        \BinaryInfC{$\step{\force{A}}{A}$}
    \end{prooftree}

%    \hfill\begin{minipage}{0.3\linewidth}  
      \begin{prooftree}
        \AxiomC{} % If we're putting these side by side we need \strut here to get rules aligned
        % \RightLabel{\textsf{error}}
        \UnaryInfC{$\step{\ctxsubst{f}{\errorU}}{\errorU}$}
      \end{prooftree}
%    \end{minipage}
%    \begin{minipage}{0.3\linewidth}
    \begin{prooftree}
        \AxiomC{$\step{M}{M'}$}  % Need \strut for side-by-side alignment again
        \UnaryInfC{$\step{\ctxsubst{f}{M}}{\ctxsubst{f}{M'}}$}
    \end{prooftree}
% \end{minipage}\hfill\hfill %% Don't know why we need two \hfills here but only one at the start
% \\
    \caption{Reduction via Contextual Semantics} %% Oops
    \label{fig:untyped-reduction}
\end{subfigure}

\bigskip

\begin{subfigure}[c]{\linewidth}
$$ \mathsf{Eval}(b, [V_1, \ldots, V_n]) \equiv \left\{
   \begin{array}{ll}
      \errorU  & \mbox{if $\denote{b}(\denote{V_1}, \ldots, \denote{V_n})= \errorX$}\\  
      \reify{\denote{b}(\denote{V_1}, \ldots, \denote{V_n})} & \mbox{otherwise}
   \end{array}
   \right. \\
$$
    \caption{Built-in function application}
    \label{fig:bif-appl}
\end{subfigure}

\caption{Term reduction for Plutus Core}
\label{fig:untyped-term-reduction}
\end{figure}

\bigskip
\noindent It can be shown that any closed Plutus Core term whose evaluation terminates
yields either \texttt{(error)} or a value.

\kwxm{I was worried because we only have rules for eg application of a builtin
  $b$ to a final argument, and when we're applying $b$ to other arguments we
  don't check that a term argument is actually expected (rather than a
  \texttt{force}), and that the argument has the right type.  I think this is OK
  though: for example, if we have a builtin $b$ with $\arity{b} = [\forall a_\#,
    \texttt{int}]$ and we have a term $M = \texttt{[(builtin b) (con 5)]}$ then
  none of the rules apply because $M$ isn't in $A$, so the semantics get stuck.
  This happens in general as well: the definition of $A$ doesn't even let us
  talk about partial builtin applications where the interleaving is wrong.  We
  \textit{do} need special rules for the final argument because if $M \in A$ we
  have to look at $b$ to make sure that the final argument (or force) is the
  right kind of thing.}
  
  
  % Also inputs untyped-values.tex
\section{The CEK machine}
This section contains a description of an abstract machine for efficiently
executing Plutus Core.  This is based on the CEK machine of Felleisen and
Friedman~\citep{Felleisen-CK-CEK}.

\noindent The machine alternates between two main phases: the
\textit{compute} phase ($\triangleright$), where it recurses down
the AST looking for values, saving surrounding contexts as frames (or
\textit{reduction contexts}) on a stack as it goes; and the
\textit{return} phase ($\triangleleft$), where it has obtained a value and
pops a frame off the stack to tell it how to proceed next.  In
addition there is an error state $\cekerror$ which halts execution
with an error, and a halting state $\cekhalt{}$ which halts execution and
returns a value to the outside world.

To evaluate a program $\texttt{(program}\ v\ M \texttt{)}$, we first check that
the version number $v$ is valid, then start the machine in the state $[];[]
\triangleright M$.  It can be proved that the transitions in
Figure~\ref{fig:untyped-cek-machine} always preserve validity of states, so that
the machine can never enter a state such as $[] \triangleleft M$ or $s,
\texttt{(force \_)} \triangleleft \texttt{(lam}\ x\ A \ M\texttt{)}$ which isn't
covered by the rules.  If such a situation were to occur in an implementation
then it would indicate that the machine was incorrectly implemented or that it
was attempting to evaluate an ill-formed program (for example, one which attempts
to apply a variable to some other term).

\begin{figure}[H]
    \centering
    \[\begin{array}{lrclr}
        \textrm{Stack} & s      & ::= & f^*\\
        \textrm{CEK value} & V &  ::= & \VCon{tn}{c} \enspace | \enspace \VDelay{M}{\rho}
               \enspace| \enspace \VLamAbs{x}{M}{\rho} \enspace | \enspace \VBuiltin{b}{\repetition{V}}{\varepsilon}\\
        \textrm{Environment} & \rho & ::= & [] \enspace | \enspace \rho[x \mapsto V] \\
        \textrm{State} & \Sigma & ::= & s;\rho \compute M \enspace | \enspace s \return V  \enspace | \enspace \cekerror{} \enspace | \enspace \cekhalt{V}\\
        \textrm{Expected builtin arguments} & \varepsilon & ::= & [\iota] \enspace | \enspace \iota::\varepsilon\\
    \end{array}\] 
    \caption{Grammar of CEK machine states for Plutus Core}
    \label{fig:untyped-cek-states}
\end{figure}
\kwxm{$\varepsilon$ is the same as $\alpha$ except that we require it to be
  nonempty syntactically, whereas we put extra conditions on $\alpha$ in the
  definition of arities for builtins that make $\alpha(b)$ nonempty for all $b
  \in \Fun$.  This means that we can never have an empty $\varepsilon$ in
  $\VBuiltin{b}{\repetition{V}}{\varepsilon}$, which isn't entirely obvious.
  We'll be in trouble here if we ever have nullary builtins.}
  
\kwxm{Do we need to insist that CEK-values are well-formed, for example that
  there are enough variables in the environments to yield closed terms and that
  in $\VBuiltin{b}{\repetition{V}}{\varepsilon}$ $\varepsilon$ is a suffix of $\arity{b}$?
  Presumably the answer is no: you'd hope that a closed (and well-formed?) term
  $M$ will always yield a well-formed CEK value.}

\begin{figure}
    \centering
    \[\begin{array}{lrclr}
        \textrm{Frame} & f  & ::=   & \inForceFrame              & \textrm{force}\\
                       &    &       & \inAppLeftFrame{(M,\rho)}  & \textrm{left application}\\
                       &    &       & \inAppRightFrame{V}        & \textrm{right application}

    \end{array}\]
    \caption{Grammar of reduction frames for Plutus Core}
    \label{fig:untyped-cek-reduction-frames}
\end{figure}

Figures~\ref{fig:untyped-cek-states} and \ref{fig:untyped-cek-reduction-frames}
define some notation for \textit{states} of the CEK machine: these involve a
modified type of value adapted to the CEK machine, enivronments which bind names
to values, and a stack which stores partially evaluated terms whose evaluation
cannot proceed until some more computation has been performed (for example,
since Plutus Core is a strict language function arguments have to be reduced to
values before application takes place, and because of this a lambda term may
have to be stored on the stack while its argument is being reduced to a value).
Environments are lists of the form $\rho = [x_1 \mapsto V_1, \ldots, x_n \mapsto
  V_n]$ which grow by having new entries appended on the right; we say that
\textit{$x$ is \textit{bound} in the environment $\rho$} if $\rho$ contains an
entry of the form $x \mapsto V$, and in that case we denote by $\rho[x]$ the
value $V$ in the rightmost (ie, most recent) such entry.

To make the CEK machine fit into the built-in evaluation mechanism defined in
Section~\ref{sec:specify-builtins} we define $\Inputs = V$ and $\Con{\tn} =
\{\VCon{tn}{c} : \tn \in \Uni, c \in \denote{\tn}\}$.

The rules in Figure~\ref{fig:untyped-cek-machine} show the transitions of the
machine; if any situation arises which is not included in these transitions (for
example, if a frame $\inAppRightFrame{\VCon{tn}{c}}$ is encountered or if an
attempt is made to apply \texttt{force} to a partial builtin application which
is expecting a term argument), then the machine stops immediately in an error
state.


% Allow page break for (slightly) better placement
\begin{figure}[H]
  \begin{subfigure}[c]{\linewidth}
    \judgmentdef{$\Sigma \mapsto \Sigma^{\prime}$}{Machine takes one step from state $\Sigma$ to state $\Sigma'$}

%\hspace{-1cm}
    \begin{minipage}{\linewidth}
\begin{alignat*}{2}
 s;\rho & \compute x                                 &~\mapsto~& s \return  \rho[x] \enskip \mbox{if $x$ is bound in $\rho$}\\
 s;\rho & \compute \con{tn}{c}                       &~\mapsto~& s \return \VCon{tn}{c}\\
 s;\rho & \compute \lamU{x}{M}                       &~\mapsto~& s \return \VLamAbs{x}{M}{\rho}\\
 s;\rho & \compute \delay{M}                         &~\mapsto~& s\return \VDelay{M}{\rho}\\
 s;\rho & \compute \force{M}                         &~\mapsto~& \inForceFrame{} \cons s;\rho \compute M \\
 s;\rho & \compute \appU{M}{N}                       &~\mapsto~& \inAppLeftFrame{(N,\rho)} \cons s ;\rho \compute M\\
% No nullary builtins (yet)
 s;\rho & \compute \builtin{b}                      &~\mapsto~& s \return \VBuiltin{b}{[]}{\arity{b}}\\
 s;\rho & \compute \errorU                           &~\mapsto~& \cekerror{}\\
\\[-10pt] %% Put some vertical space between compute and return rules, but not a whole line
[] & \return V                                    &~\mapsto~& \cekhalt{V}\\
\inAppLeftFrame{(M,\rho)}  \cons s            & \return V  &~\mapsto~& \inAppRightFrame{V} \cons s;\rho \compute M\\
\inAppRightFrame{\VLamAbs{x}{M}{\rho}} \cons s   & \return V  &~\mapsto~& s;\rho[x \mapsto V] \compute M\\
\inForceFrame{} \cons s & \return \VDelay{M}{\rho}         &~\mapsto~& s;\rho \compute M\\
\inForceFrame{} \cons s & \return \VBuiltin{b}{\repetition{V}}{(\iota \cons \varepsilon)} &~\mapsto~&
                         s \return \VBuiltin{b}{\repetition{V}}{\varepsilon} \enskip \mbox{if $\iota \in \QVar$}\\
\inForceFrame{} \cons s & \return \VBuiltin{b}{\repetition{V}}{[\iota]}   &~\mapsto~&
                         \mathsf{Eval}\,(s, b, \repetition{V}) \enskip \mbox{if $\iota \in \QVar$}\\
\inAppRightFrame{\VBuiltin{b}{\repetition{V}}{(\iota \cons \varepsilon)}} \cons s & \return V &~\mapsto~&
                         s \return \VBuiltin{b}{(\repetition{V} \snoc V)}{\varepsilon} \enskip \mbox{if $V \sim \iota$}\\
\inAppRightFrame{\VBuiltin{b}{\repetition{V}}{[\iota]}} \cons s  & \return V &~\mapsto~&
                         \mathsf{Eval}\,(s, b, \repetition{V}\snoc V) \enskip \mbox{if $V \sim \iota$}\\
\end{alignat*}
\end{minipage}
    \caption{CEK machine transitions for Plutus Core}
    \label{fig:untyped-cek-transitions}
\end{subfigure}

\bigskip
  \begin{subfigure}[c]{\linewidth}
$$ \mathsf{Eval}(s, b, [V_1, \ldots, V_n]) \equiv \left\{
   \begin{array}{ll}
      \cekerror  & \mbox{if $[V_1, \ldots, V_n] \not\approx \bar{\alpha}(b)$}\\  
      \cekerror  & \mbox{if $\denote{b}(\denote{V_1}, \ldots, \denote{V_n})= \errorX$}\\  
      s \return \reify{\denote{b}(\denote{V_1}, \ldots, \denote{V_n})} & \mbox{otherwise}
   \end{array}
   \right. \\
$$
    \caption{Evaluation of built-in functions}
    \label{fig:untyped-cek-builtins}
    \end{subfigure}

  \caption{A CEK machine for Plutus Core}
\label{fig:untyped-cek-machine}
\end{figure}

\kwxm{The first $\cekerror$ case in the definition of \textsf{Eval} says that the machine
  should halt (before we even start to evaluate $b$) if \#-variables are not
  instantiated consistently as required by the signature of $b$ and the second
  case says that the machine should halt if we do call $b$ but it returns
  $\errorX$ due to some internal failure.}

% \input{tplc.tex}
\section{Typed Plutus Core}
To follow.

\begin{appendices}
\section{Built-in Types and Functions Supported in the Alonzo Release}
\label{appendix:default-builtins-alonzo}

\newcounter{note}
\newcommand{\note}[1]{
  \bigskip
  \refstepcounter{note}
  \noindent\textbf{Note \thenote. #1}
}

\newcommand{\ty}[1]{\mathtt{#1}}
\subsection{Built-in types and type operators}
\label{sec:alonzo-built-in-types}
The Alonzo release of the Cardano blockchain supports a default set of built-in
types and type operators defined in Figures~\ref{fig:alonzo-built-in-types} and
~\ref{fig:alonzo-built-in-type-operators}.  We also include concrete syntax for
these; the concrete syntax is not strictly part of the language, but may be
useful for tools working with Plutus Core.
\begin{table}[H]
  \centering
    \begin{tabular}{|l|p{6cm}|l|}
        \hline
        Type & Denotation & Concrete Syntax\\
        \hline
        \texttt{integer} &   $\mathbb{Z}$ & \texttt{-?[0-9]*}\\
        \texttt{bytestring}  & $ \byte^*$, the set of sequences of bytes or 8-bit characters. & \texttt{\#([0-9A-Fa-f][0-9A-Fa-f])*}\\
        \texttt{string} &   The set of sequences of Unicode scalar values, as defined in \href{http://www.unicode.org/versions/Unicode5.2.0/ch03.pdf#page=35}{§3.9, definition D76 of the Unicode 5.2 standard}. & See note below\\
        \texttt{bool} & \{\texttt{true, false}\} & \texttt{True | False}\\
        \texttt{unit} &  \{()\} & \texttt{()}\\
        \texttt{data} &  See below. & Not yet supported\\
        \hline
    \end{tabular}
    \caption{Atomic Types}
    \label{fig:alonzo-built-in-types}
\end{table}

\begin{table}[H]
  \centering
    \begin{tabular}{|l|p{14mm}|l|l|}
        \hline
        Operator $\mathit{op}$ & $\left|\mathit{op}\right|$  & Denotation & Concrete Syntax\\
        \hline
        \texttt{list} & 1 & $\denote{\listOf{t}} = \denote{t}^*$ & Not yet supported\\
        \texttt{pair} & 2 & $\denote{\pairOf{t_1}{t_2}} = \denote{t_1} \times \denote{t_2}$ & Not yet supported\\
        \hline
        \end{tabular}
   \caption{Type Operators}
    \label{fig:alonzo-built-in-type-operators}
\end{table}

\paragraph{Concrete syntax for strings.} Strings are represented as sequences of Unicode characters
enclosed in double quotes, and may include standard escape sequences.

\paragraph{Concrete syntax for higher-order types.} Types such as $\listOf{\ty{integer}}$
and $\pairOf{\ty{bool}}{\ty{string)}}$ are represented by application at the
type level, thus: \texttt{[(con list) (con integer)]} and\texttt{[(con pair)
    (con bool) (con string)]}.  Each higher-order type will need further syntax
for representing constants of those types.  For example, we might use
\texttt{[]} for list values and \texttt{(,)} for pairs, so the list [11,22,33]
might be written as
\begin{verbatim}
   (con [(con list) (con integer)] 
        [(con integer 11), (con integer 22), (con integer 33)]
   )
\end{verbatim}
and the pair (True, "Plutus") as
\begin{verbatim}
   (con [(con pair) (con bool) (con string)] 
        ((con bool True), (con string "Plutus"))
   ).
\end{verbatim}
Note however that this syntax is not currently supported by most Plutus Core tools at the time of writing.



\paragraph{The $\ty{data}$ type.}
We provide a built-in type $\ty{data}$ which permits the encoding of simple data structures
for use as arguments to Plutus Core scripts.  This type is defined in Haskell as 
\begin{alltt}
data Data =
      Constr Integer [Data]
    | Map [(Data, Data)]
    | List [Data]
    | I Integer
    | B ByteString
\end{alltt}



\noindent In set-theoretic terms the denotation of $\ty{data}$ is
defined to be the least fixed point of the endofunctor $F$ on the category of
sets given by $F(X) = (\denote{\ty{integer}} \times X^*) \disj (X \times X)^* \disj
X^* \disj \denote{\ty{integer}} \disj \denote{\ty{bytestring}}$, so that
$$ \denote{\ty{data}} = (\denote{\ty{integer}} \times \denote{\ty{data}}^*)
               \disj (\denote{\ty{data}} \times \denote{\ty{data}})^*
               \disj \denote{\ty{data}}^*
               \disj \denote{\ty{integer}}
               \disj \denote{\ty{bytestring}}.
$$
We have injections
\begin{align*}
  \inj_C: \denote{\ty{integer}} \times \denote{\ty{data}}^* & \to \denote{\ty{data}} \\
  \inj_M: \denote{\ty{data}} \times \denote{\ty{data}}^*  & \to \denote{\ty{data}} \\
  \inj_L: \denote{\ty{data}}^* & \to \denote{\ty{data}} \\
  \inj_I: \denote{\ty{integer}} & \to \denote{\ty{data}} \\
  inj_B: \denote{\ty{bytestring}} & \to \denote{\ty{data}} \\
\end{align*}
\noindent and projections
\begin{align*}
  \proj_C: \denote{\ty{data}} & \to \withError{(\denote{\ty{integer}} \times \denote{\ty{data}}^*)}\\
  \proj_M: \denote{\ty{data}} & \to \withError{(\denote{\ty{data}} \times \denote{\ty{data}}^*)}\\
  \proj_L: \denote{\ty{data}} & \to \withError{\denote{\ty{data}}^* }\\
  \proj_I: \denote{\ty{data}} & \to \withError{\denote{\ty{integer}}}\\
  \proj_B: \denote{\ty{data}} & \to \withError{\denote{\ty{bytestring}} }\\
\end{align*}
\noindent which extract an object of the relevant type from a $\ty{data}$ object
$D$, returning $\errorX$ if $D$ does not lie in the expected component of the
disjoint union; also there are functions
$$
\is_C, \is_M, \is_L, \is_I, \is_B: \denote{\ty{data}} \to \denote{\ty{bool}}
$$
\noindent which determine whether a $\ty{data}$ value lies in the relevant component.


\subsection{Built-in functions}
The default set of built-in functions for the Alonzo release is shown in Figure~\ref{fig:alonzo-built-in-functions}.
\setlength{\LTleft}{-18mm}  % Shift the table left a bit to centre it on the page
\begin{longtable}[H]{|l|p{5cm}|p{5cm}|c|c|}
    \hline
    \text{Function} & \text{Signature} & \text{Denotation} & \text{Can} & \text{Note} \\
    & & & Fail? & \\
    \hline
    \endfirsthead
    \hline
    \text{Function} & \text{Type} & \text{Denotation} & \text{Can} & \text{Note}\\
    & & & Fail? & \\
    \hline
    \endhead
    \hline
    \caption{Built-in Functions}
    % This caption goes on every page of the table except the last.  Ideally it
    % would appear only on the first page and all the rest would say
    % (continued). Unfortunately it doesn't seem to be easy to do that in a
    % longtable.
    \endfoot
    \caption[]{Built-in Functions (continued)}
    \label{fig:alonzo-built-in-functions}
    \endlastfoot
    \T{addInteger}               & $[\ty{integer}, \ty{integer}] \to \ty{integer}$   & $+$ &  & \\
    \T{subtractInteger}          & $[\ty{integer}, \ty{integer}] \to \ty{integer}$   & $-$ &  & \\
    \T{multiplyInteger}          & $[\ty{integer}, \ty{integer}] \to \ty{integer}$   & $\times$ &  & \\
    \T{divideInteger}            & $[\ty{integer}, \ty{integer}] \to \ty{integer}$   & $\divfn$   & Yes & \ref{note:integer-division-functions}\\
    \T{modInteger}               & $[\ty{integer}, \ty{integer}] \to \ty{integer}$   & $\modfn$   & Yes & \ref{note:integer-division-functions}\\
    \T{quotientInteger}          & $[\ty{integer}, \ty{integer}] \to \ty{integer}$   & $\quotfn$  & Yes & \ref{note:integer-division-functions}\\
    \T{remainderInteger}         & $[\ty{integer}, \ty{integer}] \to \ty{integer}$   & $\remfn$   & Yes & \ref{note:integer-division-functions}\\
    \T{equalsInteger}            & $[\ty{integer}, \ty{integer}] \to \ty{bool}$      & $=$ &  & \\
    \T{lessThanInteger}          & $[\ty{integer}, \ty{integer}] \to \ty{bool}$      & $<$ &  & \\
    \T{lessThanEqualsInteger}    & $[\ty{integer}, \ty{integer}] \to \ty{bool}$      & $\leq$ &  & \\
    %% Some of the signatures look like $ ... $ \mbox{\;\; $ ... $} to allow a break with some indentation afterwards
    \T{appendByteString}         & $[\ty{bytestring}, \ty{bytestring}] $ \mbox{$\;\; \to \ty{bytestring}$}
                                           & $([c_1, \dots, c_m], [d_1, \ldots, d_n]) $ \mbox{$\;\; \mapsto [c_1,\ldots, c_m,d_1, \ldots, d_n]$} &  & \\
    \T{consByteString}         & $[\ty{integer}, \ty{bytestring}] $ \mbox{$\;\; \to \ty{bytestring}$}
                                          & $(c,[c_1,\ldots,c_n]) $ \mbox{$\;\;\mapsto [\text{mod}(c,256) ,c_1,\ldots,c_{n}]$} &  & \\
    \T{sliceByteString}        & $[\ty{integer}, \ty{integer}, \ty{bytestring]} $  \mbox {$\;\; \to  \ty{bytestring}$}
                                                   &   $(s,k,[c_0,\ldots,c_n])$ \mbox{$\;\;\mapsto [c_{\max(s,0)},\ldots,c_{\min(s+k-1,n-1)}]$}
                                                   &  & \ref{note:slicebytestring}\\
    \T{lengthOfByteString}       & $[\ty{bytestring}] \to \ty{integer}$ & $[] \mapsto 0, [c_1,\ldots, c_n] \mapsto n$ &  & \\
    \T{indexByteString}          & $[\ty{bytestring}, \ty{integer}] $ \mbox{$\;\; \to \ty{integer}$}
                                                   & $([c_0,\ldots,c_{n-1}],j)$ \mbox{$\;\;\mapsto
                                                       \left\{ \begin{array}{ll}
                                                         c_i & \mbox{if $0 \leq j \leq n-1$} \\
                                                         \errorX & \mbox{otherwise}
                                                       \end{array}\right.$} & Yes & \\
    \T{equalsByteString}         & $[\ty{bytestring}, \ty{bytestring}] $ \mbox{$\;\; \to \ty{bool}$}   & = &  & \ref{note:bytestring-comparison}\\
    \T{lessThanByteString}       & $[\ty{bytestring}, \ty{bytestring}] $ \mbox{$\;\; \to \ty{bool}$}   & $<$ &  & \ref{note:bytestring-comparison}\\
    \T{lessThanEqualsByteString} & $[\ty{bytestring}, \ty{bytestring}] $ \mbox{$\;\; \to \ty{bool}$}   & $\leq$ &  & \ref{note:bytestring-comparison}\\
    \T{appendString}             & $[\ty{string}, \ty{string}] \to \ty{string}$
                                         & $([u_1, \dots, u_m], [v_1, \ldots, v_n]) $ \mbox{$\;\; \mapsto [u_1,\ldots, u_m,v_1, \ldots, v_n]$} &  & \\
    \T{equalsString}             & $[\ty{string}, \ty{string}] \to \ty{bool}$           & = &  & \\
    \T{encodeUtf8}               & $[\ty{string}] \to \ty{bytestring}$      & Convert a $\ty{string}$ to a \texttt{byte\-string}. & &
                                                                                                               \ref{note:bytestring-encoding} \\
    \T{decodeUtf8}               & $[\ty{bytestring}] \to \ty{string}$      & Convert a $\ty{bytestring}$ to a $\ty{string}$. & Yes
                                                                                                                  & \ref{note:bytestring-encoding} \\
    \T{sha2\_256}                & $[\ty{bytestring}] \to \ty{bytestring}$  & Hash a $\ty{bytestring}$ using \T{SHA\-256}. &  & \\
    \T{sha3\_256}                & $[\ty{bytestring}] \to \ty{bytestring}$  & Hash a $\ty{bytestring}$ using \T{SHA3\-256}. &  & \\
    \T{blake2b\_256}             & $[\ty{bytestring}] \to \ty{bytestring}$  & Hash a $\ty{bytestring}$ using \T{Blake2B\-256}. &  & \\
    \T{verifySignature}          & $[\ty{bytestring}, \ty{bytestring}, $ \mbox{$\;\; \ty{bytestring}] \to \ty{bool}$}
                                                  & Verify the signature using \T{Ed25519}. &  Yes & \ref{note:signature-verification}\\
    \T{ifThenElse}               & $[\forall a_*, \ty{bool}, a_*, a_*] \to a_*$
                                                 & \mbox{$(\mathtt{true},t_1,t_2) \mapsto t_1$}
                                                 \mbox{$(\mathtt{false},t_1,t_2) \mapsto t_2$} & & \\
    \T{chooseUnit}               & $[\forall a_*, \ty{unit}, a_*] \to a_*$        & $((), t) \mapsto t$ & & \\
    \T{trace}                    & $[\forall a_*, \ty{string}, a_*] \to a_*$      & $ (s,t) \mapsto t$ &  & \ref{note:trace}\\
    \T{fstPair}                  & $[\forall a_\#, \forall b_\#, \pairOf{a_\#}{b_\#}] \to a_\#$       & $(x,y) \mapsto x$ && \\
    \T{sndPair}                  & $[\forall a_\#, \forall b_\#, \pairOf{a_\#}{b_\#}] \to b_\#$       & $(x,y) \mapsto y$ & & \\
    \T{chooseList}               & $[\forall a_\#, \forall b_*, \listOf{a_\#}, b_*, b_*] \to b_*$
                                              & \mbox{$([], t_1, t_2) \mapsto t_1$,} \mbox{$([x_1,\ldots,x_n],t_1,t_2) \mapsto t_2\ (n \geq 1)$}. & & \\
    \T{mkCons}                   & $[\forall a_\#, a_\#, \listOf{a_\#}] \to \listOf{a _\#}$  & $(x,[x_1,\ldots,x_n]) \mapsto [x,x_1,\ldots,x_n]$ & Yes & \\
    \T{headList}                 & $[\forall a_\#, \listOf{a_\#}] \to a_\#$               & $[]\mapsto \errorX, [x_1,x_2, \ldots, x_n] \mapsto x_1$ & Yes & \\
    \T{tailList}                 & $[\forall a_\#, \listOf{a_\#}] \to \listOf{a_\#}$
                                        &  \mbox{$[] \mapsto \errorX$,} \mbox{$ [x_1,x_2, \ldots, x_n] \mapsto [x_2, \ldots, x_n]$} & Yes & \\
    \T{nullList}                 & $[\forall a_\#, \listOf{a_\#}] \to \ty{bool}$            & $ [] \mapsto \T{true},
                                                                                                    [x_1,\ldots, x_n] \mapsto \T{false}$& & \\
    \T{chooseData}               & $[\forall a_*, \ty{data}, a_*, a_*, a_*, a_*, a_*] \to a_*$
    & $ (d,t_C, t_M, t_L, t_I, t_B) $
    \smallskip
    \newline  % The big \{ was abutting the text above
    \mbox{$\;\;\mapsto
                                                       \left\{ \begin{array}{ll}
                                                         t_C  & \mbox{if $\is_C(d)$} \\
                                                         t_M  & \mbox{if $\is_M(d)$} \\
                                                         t_L  & \mbox{if $\is_L(d)$} \\
                                                         t_I  & \mbox{if $\is_I(d)$} \\
                                                         t_B  & \mbox{if $\is_B(d)$} \\
                                                       \end{array}\right.$}  & & \\
    \T{constrData}               & $[\ty{integer}, \listOf{\ty{data}}] \to \ty{data}$          & $\inj_C$ & & \\
    \T{mapData}                  & $[\listOf{\pairOf{\ty{data}}{\ty{data}}}$ \mbox{$\;\; \to \ty{data}$}     & $\inj_M$& & \\
    \T{listData}                 & $[\listOf{\ty{data}}] \to \ty{data} $      & $\inj_L$& & \\
    \T{iData}                    & $[\ty{integer}] \to \ty{data} $            & $\inj_I$ & & \\
    \T{bData}                    & $[\ty{bytestring}] \to \ty{data} $         & $\inj_B$& & \\
    \T{unConstrData}             & $[\ty{data}] \to \pairOf{\ty{integer}}{\ty{data}} $            & $\proj_C$ & Yes& \\
    \T{unMapData}                & $[\ty{data}] $\mbox{$\;\; \to \listOf{\pairOf{\ty{data}}{\ty{data}}}$}  & $\proj_M$ & Yes& \\
    \T{unListData}               & $[\ty{data}] \to \listOf{\ty{data}} $                          & $\proj_L$ & Yes& \\
    \T{unIData}                  & $[\ty{data}] \to \ty{integer} $                                & $\proj_I$ & Yes& \\
    \T{unBData}                  & $[\ty{data}] \to \ty{bytestring} $                             & $\proj_B$ & Yes& \\
    \T{equalsData}               & $[\ty{data}, \ty{data}] \to \ty{bool} $                        & $ = $ & & \\
    \T{mkPairData}               & $[\ty{data}, \ty{data}]$ \mbox{\;\; $\to \pairOf{\ty{data}}{\ty{data}}$}  & $(x,y) \mapsto (x,y) $ & & \\
    \T{mkNilData}                & $[\ty{unit}] \to \listOf{\ty{data}} $                       & $() \mapsto []$ & & \\
    \T{mkNilPairData}            & $[\ty{unit}] $ \mbox{$\;\; \to \listOf{\pairOf{\ty{data}}{\ty{data}}} $}   & $() \mapsto []$ & & \\
    \hline 
\end{longtable}

\kwxm{Maybe try \texttt{tabulararray} to see what sort of output that gives for the big table.}

\note{Integer division functions.}
\label{note:integer-division-functions}
We provide four integer division functions: \texttt{divideInteger},
\texttt{modInteger}, \texttt{quotientInteger}, and \texttt{remainderInteger},
whose denotations are mathematical functions $\divfn, \modfn, \quotfn$, and
$\remfn$ which are modelled on the corresponding Haskell operations. Each of
these takes two arguments and will fail (returning $\errorX$) if the second one
is negative.  For all $a,b \in \Z$ with $a \ne 0$ we have
$$
\divfn(a,b) \times b + \modfn(a,b) = a
$$
$$
  |\modfn(a,b)| < |b|
$$\noindent and
$$
  \quotfn(a,b) \times b + \remfn(a,b) = a
$$
$$
  |\remfn(a,b)| < |b|.
$$
\noindent The $\divfn$ and $\modfn$ functions form a pair, as do $\quotfn$ and $\remfn$;
$\divfn$ should not be used in combination with $\modfn$, not should $\quotfn$ be used
with $\modfn$.

For positive divisors $b$, $\divfn$ truncates downwards and $\modfn$ always
returns a non-negative result ($0 \leq \modfn(a,b) \leq b-1$).  The $\quotfn$
function truncates towards zero.  Figure~\ref{fig:integer-division-signs} shows
how the signs of the outputs of the division functions depend on the signs of
the inputs ($+$ means non-negative, so includes 0).
\begin{table}[H]
  \centering
    \begin{tabular}{|cc|cc|cc|}
        \hline
        a & b & $\divfn$ & $\modfn$ & $\quotfn$ & $\remfn$ \\
        \hline
        $+$ & $+$ & $+$ & $+$ & $+$ & $+$ \\
        $-$ & $+$ & $-$ & $+$ & $-$ & $-$ \\
        $+$ & $-$ & $-$ & $+$ & $+$ & $+$ \\
        $-$ & $-$ & $+$ & $-$ & $+$ & $-$ \\
        \hline
        \end{tabular}
   \caption{Behaviour of integer division functions}
   \label{fig:integer-division-signs}
\end{table}
%% -------------------------------
%% |   n  d | div mod | quot rem |
%% |-----------------------------|
%% |  41  5 |  8   1  |   8   1  |
%% | -41  5 | -9   4  |  -8  -1  |
%% |  41 -5 | -9  -4  |  -8   1  |
%% | -41 -5 |  8  -1  |   8  -1  |
%% -------------------------------

\note{The \texttt{sliceByteString} function.}
\label{note:slicebytestring}
The application \texttt{[[(builtin sliceByteString) (con integer $s$)] (con
    integer $k$)] (con bytestring $b$)]} returns the substring of $b$ of length
$k$ starting at position $s$; indexing is zero-based, so a call with $s=0$
returns a substring starting with the first element of $b$, $s=1$ returns a
substring starting with the second, and so on.  This function always succeeds,
even if the arguments are out of range: if $b=[c_0, \ldots, c_{n-1}]$ then the
  application above returns the substring $[c_i, \ldots, c_j]$ where
  $i=\max(s,0)$ and $j=\min(s+k-1, n-1)$; if $j<i$ then the empty string is returned.
  

\note{Comparisons of bytestrings.}
\label{note:bytestring-comparison}
Bytestrings are ordered lexicographically.  If we have $a = [c_1, \ldots, c_m]$
and $b = [d_1, \ldots, d_n]$ then
\begin{itemize}
\item $a = b$ if and only if $m=n$ and $c_i = d_i$ for $1 \leq i \leq m$
\item $a \leq b$ if $c_i = d_i$ for $1 \leq i \leq \min(m,n)$
\item $a<b$ if $a \leq b$ and $a \neq b$.
\end{itemize}
\noindent The empty bytestring is equal only to itself and is strictly less than all other bytestrings.

\kwxm{The lexicographic ordering means that $\mathtt{\#23456789} <
  \mathtt{\#24}$, which came as a slight surprise to me.  I think I was thinking
  of these as long numbers, not strings.}

\note{Encoding and decoding bytestrings.}
\label{note:bytestring-encoding}
The \texttt{encodeUtf8} and \texttt{decodeUtf8} functions convert between the
$\ty{string}$ type and the $\ty{bytestring}$ type.  We have defined
$\denote{\ty{string}}$ to consist of sequences of Unicode characters without
specifying any particular character representation, but
$\denote{\ty{bytestring}}$ consists of sequences of 8-bit bytes.  As the names
suggest, both functions use the well-known UTF-8 character encoding, where each
Unicode character is encoded using between one and four bytes: thus in general
neither function will preserve the length of an object; moreover, not all
sequences of bytes are valid representations of Unicode characters, and
\texttt{decodeUtf8} will fail if it receives an invalid input (but
\texttt{encodeUtf8} will always succeed).

\kwxm{In fact, strings are represented as sequences of UTF-16 characters, which
  use two or four bytes per character.  Do we need to mention that?  If we do,
  we'll need to be a little careful: there are sequences of 16-bit words that
  don't represent valid Unicode characters (for example, if the sequence uses
  surrogate codepoints improperly.  I don't think you can create a Haskell
  \texttt{Text} object (which is what our strings really are) that's invalid
  though.}


\note{Signature verification.}
\label{note:signature-verification}
The \texttt{verifySignature} performs cryptographic signature verification using
the Ed25519 scheme (\cite{ches-2011-24091}).  The function takes three bytestring
arguments: a public key $k$, a message $m$, and a signature $s$ (in that order).
The key $k$ must be exactly 32 bytes long and the signature $s$ must be exactly 64
bytes, and the function will fail if either is the wrong length; there is no
restriction on the length of the message. If $k$ and $s$ are the correct lengths
then \texttt{verifySignature} returns \texttt{true} if the private key
corresponding to $k$ was used to sign the message $m$ to produce $s$, otherwise
it returns \texttt{false}.

\note{The \texttt{trace} function.}
\label{note:trace}
An application \texttt{[(builtin trace) $s$ $v$]} ($s$ a \texttt{string}, $v$
any Plutus Core value) returns $v$.  We do not specify the semantics any
further.  An implementation may choose to discard $s$ or to perform some
side-effect such as writing it to a terminal or log file.

\kwxm{I sincerely hope that this is the only function we'll ever have that can
  do things like this.  I don't want to have to introduce some general notion of
  effectfulness into the notation to deal with the general case.}

\subsection{Cost accounting for built-in functions}
To follow.

\newpage

\section{Formally Verified Behaviours}
To follow.

\section{A Binary Serialisation Format for Plutus Core Terms and Programs}
\label{appendix:serialisation}

We use the \texttt{flat} \citep{flat} format to serialise Plutus Core terms. The
\texttt{flat} format encodes sum types as tagged unions and products by
concatenating their contents. We proceed by defining the structure and the data
types of untyped Plutus Core and how they get serialised.

\subsection{Variable length data}

\texttt{Non-empty lists} are encoded by prefixing the element stored with `0'
if this is the \texttt{last} element or `1' if there is \texttt{more} data following.

\noindent We encode \texttt{Integers} as a non-empty list of chunks, 7 bits each,
with the least significant chunk first and the most significant bit first in the chunk.

\medskip
\noindent Let's calculate the encoding of the \texttt{32768} index (unsigned, arbitrary
length integer):
\begin{enumerate}
  \item Converting \texttt{32768} to binary: \\
    \verb|32768| $\rightarrow$ \verb|0b1000000000000000|
  \item Split into 7 bit chunks: \\
    \verb|0b1000000000000000| $\rightarrow$ \verb|0000010 0000000 0000000|
  \item Reorder chunks (least significant chunk first): \\
    \verb|0000010 0000000 0000000| $\rightarrow$ \verb|0000000 0000000 0000010|
  \item Add list constructor tags: \\
    \verb|0000000 0000000 0000010| $\rightarrow$ \verb|10000000 10000000 00000010|
\end{enumerate}

For \texttt{ByteString}s and \texttt{String}s we use a byte aligned array of
bytes (in the case of \texttt{String} the bytes correspond to the UTF-8 encoding
of the text). The structure is pre-aligned to the byte boundary by using the `0'
bit as a filler and the `1' bit as the final bit. Following the filler we have
the number of bytes that the data uses, a number from 0 to 255 (1 byte),
followed by the bytes themselves, and a final 0 length block (the byte `0').

\kwxm{I was worried that we were maybe using the UTF-16 enconding for strings,
  because internally they're \texttt{Text} objects, which are arrays of UTF-16
  things.  But \texttt{flat} encodes \texttt{Text} using UTF-8: see
  \url{https://hackage.haskell.org/package/flat-0.4.4/docs/Flat-Instances-Text.html}.}

\subsection{Constants}
Constants are encoded as a combination of a sequence of 4-bit tags indicating
the type of value that is serialised and the value itself: see
Figure~\ref{fig:serialisation-constants}. Constants use 4 bits to encode the
type tags, so they allow for a maximum of 16 constructors, of which 9 are used
by the default set of builtin types.

\vspace{1cm}

\begin{minipage}{\linewidth}
\centering
\begin{tabular}{|l|c|l|}
  \hline
  \Strut
  \textrm{Name} & \textrm{Tag} & \textrm{Encoding} \\
  \hline
  \T{$\ty{integer}$}    & 0 & ZigZag + Variable length \rule{0mm}{4mm}\\[\sep]
  \T{$\ty{bytestring}$} & 1 & Variable length \\[\sep]
  \T{$\ty{string}$}     & 2 & UTF-8 \\[\sep]
  \T{$\ty{unit}$}       & 3 & Empty \\[\sep]
  \T{$\ty{bool}$}       & 4 & `1' is True, `0' is False \\[\sep]
  \T{$\ty{list}$}       & 5 & See below \\[\sep]
  \T{$\ty{pair}$}       & 6 & See below \\[\sep]
  \T{Type application}  & 7 & See below \\[\sep]
  \T{$\ty{data}$}       & 8 & See below \\
  \hline
\end{tabular}
\captionof{figure}{Serialising constants}
\label{fig:serialisation-constants}
\end{minipage}

\paragraph{Encoding types.} Basic types (those in $\Uni_0$) and type operators (in $\TyOp$)
are encoded using a single tag.  Complex types such as $\listOf{\ty{integer}}$
or $\pairOf{\ty{bool}}{\listOf{\ty{string}}}$ are regarded as iterated
applications $(\ldots((op(t_1))(t_2)\ldots)(t_n)$ and are encoded by emitting a
special tag for each application $t(t^\prime)$ followed by the encodings of $t$ and
$t^\prime$ (this approach permits some extra flexibility which we do not currently
use).  Thus in the default set of built-in types, $\listOf{\ty{integer}}$ would
be encoded as the concatenation of the sequence of 4-bit numbers $(7,5,0)$ and
$\pairOf{\ty{bool}}{\listOf{\ty{string}}}$ would be encoded as the concatenation
of the sequence $(7,7,6,4,7,5,2)$.

% ( ... ((op(t1))(t2) ... )(t_n)
% apply (apply pair bool) (apply list string)

\paragraph{Encoding data.}
Values of type $\ty{data}$ are encoded by emitting the tag for $\ty{data}$ followed
by the CBOR encoding (see~\cite{rfc8949}) of the Haskell type
\begin{verbatim}
  data Data =
        Constr Integer [Data]
      | Map [(Data, Data)]
      | List [Data]
      | I Integer
      | B BS.ByteString
\end{verbatim}
\noindent
See the Haskell code in \texttt{plutus-core/plutus-core/src/PlutusCore/Data.hs}
in the Plutus GitHub repository~\cite{plutus-repo} for full details of this encoding. 
\kwxm{Oh dear.}


\paragraph{Encoding primitive values.}
\begin{itemize}
  \item Unsigned values of type $\ty{integer}$ and $\ty{bytestring}$ are encoded using the
    previously introduced encoding for variable length data types.
  \item Signed $\ty{integer}$ values are first converted to an unsigned value
    using the \texttt{ZigZag}\footnote{The \texttt{ZigZag} encoding interleaves
      positive and negative numbers such that small negative numbers are stored
      using a small number of bytes.} encoding, then they are encoded as
    variable length data types.
  \item \texttt{string}s are encoded as lists of characters and use the \texttt{UTF-8}
    encoding.
  \item The single \texttt{()} value of the $\ty{unit}$ type is removed from the
    serialised data, as are the constructors of any data structure which has only one constructor.
  \item Variable names are encoded using DeBruijn indices, which are unsigned, arbitrary
    length integers.
\end{itemize}
\kwxm{Check the thing about single-constructor types.  Does this mean types which have one
  nullary constructor, or do we just miss out the tag when encoding \texttt{K $x_1$ \dots $x_n$}?}

Possibly empty lists are encoded in the standard way, by the tag for the constructor,
`0' for `Nil' and `1' for `Cons'. The `Cons' constructor is followed by the serialised
element, and then, recursively another list.

Encoded values are aligned to byte/word boundary using a meaningless sequence of `0' bits
terminated with a `1' bit.

\kwxm{Should we say something about the 64-byte on-chain limit for certain things?}

\subsection{Untyped terms}

Terms are encoded using 4 bit tags, which allows a total of 16 kinds of term, of which 8 are
currently used.

\vspace{1cm}

\begin{minipage}{\linewidth}
\centering
\begin{tabular}{|l|c|l|}
  \hline
  \Strut
  \textrm{Name} & \textrm{Tag} & \textrm{Arguments} \\
  \hline
  Variable & 0 & name \rule{0mm}{4mm} \\[\sep]
  Delay & 1 & term \\[\sep]
  Lambda abstraction & 2 & name, term \\[\sep]
  Application & 3 & term, term \\[\sep]
  Constant & 4 & constant \\[\sep]
  Force & 5 & term \\[\sep]
  Error & 6 & term \\[\sep]
  Builtin & 7 & builtin \\[\sep]
  \hline
\end{tabular}
\captionof{figure}{Untyped terms}
\label{fig:serialisation-terms}
\end{minipage}

\vspace{1cm}

\subsection{Built-in functions}
Built-in functions use 8 bits for their tags, allowing for a maximum of 128 builtin
functions of which 51 are currently used.
\vspace{1cm}

\begin{minipage}{\linewidth}
\centering
\begin{tabular}{|l|c|l|c|l|c|}
  \hline
  \Strut
  \textrm{Name} & \textrm{Tag} & \textrm{Name} & \textrm{Tag} & \textrm{Name} & \textrm{Tag} \\
  \hline
   \T{addInteger}               &    0    &     \T{blake2b\_256}             &   20    &   \T{iData}                    &   40    \rule{0mm}{4mm} \\[\sep]
   \T{subtractInteger}          &    1    &     \T{verifySignature}          &   21    &   \T{bData}                    &   41    \\[\sep]
   \T{multiplyInteger}          &    2    &     \T{appendString}             &   22    &   \T{unConstrData}             &   42    \\[\sep]
   \T{divideInteger}            &    3    &     \T{equalsString}             &   23    &   \T{unMapData}                &   43    \\[\sep]
   \T{quotientInteger}          &    4    &     \T{encodeUtf8}               &   24    &   \T{unListData}               &   44    \\[\sep]
   \T{remainderInteger}         &    5    &     \T{decodeUtf8}               &   25    &   \T{unIData}                  &   45    \\[\sep]
   \T{modInteger}               &    6    &     \T{ifThenElse}               &   26    &   \T{unBData}                  &   46    \\[\sep]
   \T{equalsInteger}            &    7    &     \T{chooseUnit}               &   27    &   \T{equalsData}               &   47    \\[\sep]
   \T{lessThanInteger}          &    8    &     \T{trace}                    &   28    &   \T{mkPairData}               &   48    \\[\sep]
   \T{lessThanEqualsInteger}    &    9    &     \T{fstPair}                  &   29    &   \T{mkNilData}                &   49    \\[\sep]
   \T{appendByteString}         &   10    &     \T{sndPair}                  &   30    &   \T{MkNilPairData}            &   50    \\[\sep]
   \T{consByteString}           &   11    &     \T{chooseList}               &   31    & & \\[\sep]
   \T{sliceByteString}          &   12    &     \T{mkCons}                   &   32    & & \\[\sep]
   \T{lengthOfByteString}       &   13    &     \T{headList}                 &   33    & & \\[\sep]
   \T{indexByteString}          &   14    &     \T{tailList}                 &   34    & & \\[\sep]
   \T{equalsByteString}         &   15    &     \T{nullList}                 &   35    & & \\[\sep]
   \T{lessThanByteString}       &   16    &     \T{chooseData}               &   36    & & \\[\sep]
   \T{lessThanEqualsByteString} &   17    &     \T{constrData}               &   37    & & \\[\sep]
   \T{sha2\_256}                &   18    &     \T{mapData}                  &   38    & & \\[\sep]
   \T{sha3\_256}                &   19    &     \T{listData}                 &   39    & & \\[\sep]
   \hline
\end{tabular}
\captionof{figure}{Builtin tags}
\label{fig:serialisation-builtins}
\end{minipage}

\vspace{1cm}

\subsection{Example}

We will serialise the program \verb|(program 11.22.33 (con integer 11))| compiled to untyped Plutus Core, using DeBruijn indices.

First, lets convert the program to the desired representation:

\begin{verbatim}
> stack exec plc -- convert --untyped --if plc --of flat -o program.flat <<EOF
> (program 11.22.33 (con integer 11))
> EOF
\end{verbatim}

Now, let's take a look at the output.

\begin{verbatim}
> xxd -b program.flat
> 00000000: 00001011 00010110 00100001 01001000 00000101 10000001  ..!H..
\end{verbatim}

\subsubsection{The program preamble.}

We define `Program` in the `PlutusCore.Core.Type` haskell module like this:

\begin{verbatim}
-- | A 'Program' is simply a 'Term' coupled with a 'Version' 
--   of the core language.
data Program tyname name uni fun ann = 
       Program ann (Version ann) (Term tyname name uni fun ann)
         deriving (Show, Functor, Generic, NFData, Hashable)
\end{verbatim}

Because the \verb|Program| data type has only one constructor we know that flat will not waste any space serialising it. `ann' will always be (for serialised ASTs) `()', which similarly to the \verb|Program| data type has only one constructor and flat will not serialise it.

Next, the `Version' is a tuple of 3 `Natural' numbers, which are encoded as variable length unsigned integers. Because all the version numbers can fit in a 7 bit word, we only need one byte to store each of them. Also, the first bit, which represents the non-empty list constructor will always be `0' (standing for `Last'), resulting in:

\begin{verbatim}
0 (*Last*) 000 (*Unused*) 1011   (*11 in binary*)
0 (*Last*) 00  (*Unused*) 10110  (*22 in binary*)
0 (*Last*) 0   (*Unused*) 100001 (*33 in binary*)
\end{verbatim}

\subsubsection{The integer constant `1`.}

Let's take a quick look at how we defined untyped Plutus Core terms, in Figure~\ref{fig:serialisation-terms}.

We need to encode the `Constant', signed integer value `11'. Terms are encoded using 4 bits, and the `Constant' term has tag 4. This results in:

\begin{verbatim}
0 (*Unused*) 100 (*4 in binary*)
\end{verbatim}

For the `Default' universe we have the constant tags defined from Figure~\ref{fig:serialisation-constants}, wrapped in a list, followed by the encoding for the constant's value.

So we see how, for the integer type we care about the type is encoded as a list containing the id `0'. We know that we are using 3 bits to store the type of constant, so the encoding will be:

\begin{verbatim}
1 (*Cons*) 0000 (The `0` tag using 4 bits for storage) 0 (*Nil*)
\end{verbatim}

The annotation will not be serialised, and we are left with the constant itself. Because it is an variable length signed integer, we first need to find out it's value after conversion to the `ZigZag' format.

\begin{verbatim}
> stack repl plutus-core:exe:plc
> ghci> import Data.ZigZag
> ghci> zigZag (11 :: Integer)
> 22
\end{verbatim}

Next, we need to encode the variable length unsigned integer `22'. We only need one byte (as it fits in the available 7 bits), so we end up with the following:

\begin{verbatim}
0 (*Last*) 00 (*Unused*) 10110 (*22 in binary*) 000001 (*Padding to byte size*)
\end{verbatim}

\subsubsection{Note}

You may notice how in the rest of the codebase we use the `CBOR' format to serialise
everything.

So why did we choose to switch to `Flat' for on-chain serialisation?

`CBOR' pays a price for being a self-describing format. The size of the serialised
terms is consistently larger than a format that is not self-describing. Running the
`flat' benchmarks will show flat consistently out-performing `CBOR' by about 35\%
without using compression.

\begin{verbatim}
> stack bench plutus-benchmark:flat
> cat plutus-benchmark/flat-sizes.md

** Contract: crowdfunding-indices **
Codec            Size    Of minimum   Of maximum
flat             8148    2.240308     0.62652826
cbor             13005   3.5757492    1.0

** Contract: escrow-indices **
Codec            Size    Of minimum   Of maximum
flat             8529    2.2004645    0.6302838 
cbor             13532   3.491228     1.0

** Contract: future-indices **
Codec            Size    Of minimum   Of maximum
flat             17654   2.19141      0.6628619 
cbor             26633   3.305983     1.0

** Contract: game-indices **
Codec            Size   Of minimum   Of maximum
flat             5158   2.2290406    0.6254395 
cbor             8247   3.5639584    1.0

** Contract: vesting-indices **
Codec            Size    Of minimum   Of maximum
flat             8367    2.2288227    0.6273525 
cbor             13337   3.5527437    1.0
\end{verbatim}

\end{appendices}

\bibliographystyle{plainnat} %% ... or whatever
\bibliography{plutus-core-specification}

\end{document}
