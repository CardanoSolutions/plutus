\section{The Grammar of Plutus Core}
\label{sec:untyped-plc-grammar}
This section presents the grammar of Plutus Core in a Lisp-like form.  This is
intended as a specification of the abstract syntax of the language; it may also
by used by tools as a concrete syntax for working with Plutus Core programs, but
this is a secondary use and we do not make any guarantees of its completeness
when used in this way.  The primary concrete form of Plutus Core programs is the
binary format described in Appendix~\ref{appendix:serialisation}.

\subsection{Lexical grammar}
\label{sec:untyped-plc}
\thispagestyle{plain}
\pagestyle{plain}

\begin{minipage}{\linewidth}
    \centering
    \[\begin{array}{lrclr}

        \textrm{Name}        & n      & ::= & \texttt{[a-zA-Z][a-zA-Z0-9\_\textquotesingle]\textsuperscript{*}}   & \textrm{name}\\

        \textrm{Var}           & x      & ::= & n & \textrm{term variable}\\
        \textrm{BuiltinName}   & bn     & ::= & n & \textrm{built-in function name}\\
        \textrm{Version} & v & ::= & \texttt{[0-9]\textsuperscript{+}.[0-9]\textsuperscript{+}.[0-9]\textsuperscript{+}}& \textrm{version}\\

        \textrm{Constant} & c & ::= & \langle{\textrm{literal constant}}\rangle& \\

    \end{array}\]
    \captionof{figure}{Lexical grammar of Plutus Core}
    \label{fig:lexical-grammar-untyped}
\end{minipage}



%   @sqs   = '  ( ($printable # ['\\])  | (\\$printable) )* '
%   
%   -- A double quoted string, allowing escaped characters including \".  Similar to @sqs
%   @dqs   = \" ( ($printable # [\"\\]) | (\\$printable) )* \"
%   
%   -- A sequence of printable characters not containing '(' or ')' such that the
%   -- first character is not a space or a single or double quote.  If there are any
%   -- further characters then they must comprise a sequence of printable characters
%   -- possibly including spaces, followed by a non-space character.  If there are
%   -- any leading or trailing spaces they will be consumed by the $white+ token
%   -- below.
%   $nonparen = $printable # [\(\)]
%   @chars = ($nonparen # ['\"$white]) ($nonparen* ($nonparen # $white))?
%   
%       <literalconst> "()" | @sqs | @dqs | @chars { tok (\p s -> alex $ TkLiteralConst p (textOf s)) `andBegin` 0 }

%% "()"
%% @sqs   = '  ( ($printable # ['\\])  | (\\$printable) )* '
%% @dqs   = \" ( ($printable # [\"\\]) | (\\$printable) )* \"
%% @chars = ($nonparen # ['\"$white]) ($nonparen* ($nonparen # $white))?


\subsection{Grammar}
\begin{minipage}{\linewidth}
    \centering
    \[\begin{array}{lrclr}
    \textrm{Term}       & L,M,N  & ::= & x                      & \textrm{variable}\\
                        &        &     & \con{tn}{c}            & \textrm{constant}\\
                        &        &     & \builtin{b}            & \textrm{builtin}\\
                        &        &     & \lamU{x}{M}            & \textrm{$\lambda$ abstraction}\\
                        &        &     & \appU{M}{N}            & \textrm{function application}\\
                        &        &     & \delay{M}              & \textrm{delay execution of a term}\\
                        &        &     & \force{M}              & \textrm{force execution of a term}\\
                        &        &     & \errorU                & \textrm{error}\\
        \textrm{Program}& P      & ::= & \version{v}{M}         & \textrm{versioned program}

    \end{array}\]
    \captionof{figure}{Grammar of untyped Plutus Core}
    \label{fig:untyped-grammar}
\end{minipage}

\paragraph{Iterated applications.}
An application of a term $M$ to a term $N$ is represented by
$\appU{M}{N}$.   We may occasionally write $\appU{M}{N_1
  \ldots N_k}$ as an abbreviation for an iterated application
$\mathtt{[}\ldots\mathtt{[[}M\;N_1\mathtt{]}\;N_2\mathtt{]}\ldots
  N_k\mathtt{]}$, and tools may also use this as concrete syntax.

\paragraph{Built-in types and functions.} The language is parameterised by a set $\Uni$ of
\textit{built-in types} (we sometimes refer to $\Uni$ as the
\textit{universe}) and a set $\Fun$ of \textit{built-in functions}
(\textit{builtins} for short), both of which are sets of Names.
Briefly, the built-in types represent sets of constants such as
integers or strings; constant expressions $\con{tn}{c}$ represent
values of the built-in types (the integer 123 or the string
\texttt{"string"}, for example), and built-in functions are functions
operating on these values, and possibly also general Plutus Core
terms.  Precise details are given in
Section~\ref{sec:specify-builtins}.  Plutus Core comes with a default
universe and a default set of builtins, which are described in
Appendix~\ref{appendix:default-builtins-alonzo}.

