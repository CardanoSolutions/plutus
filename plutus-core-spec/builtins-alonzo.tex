\section{Built-in Types and Functions Supported in the Alonzo Release}
\label{appendix:default-builtins-alonzo}

\newcounter{note}
\newcommand{\note}[1]{
  \bigskip
  \refstepcounter{note}
  \noindent\textbf{Note \thenote. #1}
}

\newcommand{\ty}[1]{\mathtt{#1}}
\subsection{Built-in types and type operators}
\label{sec:alonzo-built-in-types}
The Alonzo release of the Cardano blockchain supports a default set of built-in
types and type operators defined in Figures~\ref{fig:alonzo-built-in-types} and
~\ref{fig:alonzo-built-in-type-operators}.  We also include concrete syntax for
these; the concrete syntax is not strictly part of the language, but may be
useful for tools working with Plutus Core.
\begin{table}[H]
  \centering
    \begin{tabular}{|l|p{6cm}|l|}
        \hline
        Type & Denotation & Concrete Syntax\\
        \hline
        \texttt{integer} &   $\mathbb{Z}$ & \texttt{-?[0-9]*}\\
        \texttt{bytestring}  & $ \byte^*$, the set of sequences of bytes or 8-bit characters. & \texttt{\#([0-9A-Fa-f][0-9A-Fa-f])*}\\
        \texttt{string} &   The set of sequences of Unicode scalar values, as defined in \href{http://www.unicode.org/versions/Unicode5.2.0/ch03.pdf#page=35}{§3.9, definition D76 of the Unicode 5.2 standard}. & See note below\\
        \texttt{bool} & \{\texttt{true, false}\} & \texttt{True | False}\\
        \texttt{unit} &  \{()\} & \texttt{()}\\
        \texttt{data} &  See below. & Not yet supported\\
        \hline
    \end{tabular}
    \caption{Atomic Types}
    \label{fig:alonzo-built-in-types}
\end{table}

\begin{table}[H]
  \centering
    \begin{tabular}{|l|p{14mm}|l|l|}
        \hline
        Operator $\mathit{op}$ & $\left|\mathit{op}\right|$  & Denotation & Concrete Syntax\\
        \hline
        \texttt{list} & 1 & $\denote{\listOf{t}} = \denote{t}^*$ & Not yet supported\\
        \texttt{pair} & 2 & $\denote{\pairOf{t_1}{t_2}} = \denote{t_1} \times \denote{t_2}$ & Not yet supported\\
        \hline
        \end{tabular}
   \caption{Type Operators}
    \label{fig:alonzo-built-in-type-operators}
\end{table}

\paragraph{Concrete syntax for strings.} Strings are represented as sequences of Unicode characters
enclosed in double quotes, and may include standard escape sequences.

\paragraph{Concrete syntax for higher-order types.} Types such as $\listOf{\ty{integer}}$
and $\pairOf{\ty{bool}}{\ty{string)}}$ are represented by application at the
type level, thus: \texttt{[(con list) (con integer)]} and\texttt{[(con pair)
    (con bool) (con string)]}.  Each higher-order type will need further syntax
for representing constants of those types.  For example, we might use
\texttt{[]} for list values and \texttt{(,)} for pairs, so the list [11,22,33]
might be written as
\begin{verbatim}
   (con [(con list) (con integer)] 
        [(con integer 11), (con integer 22), (con integer 33)]
   )
\end{verbatim}
and the pair (True, "Plutus") as
\begin{verbatim}
   (con [(con pair) (con bool) (con string)] 
        ((con bool True), (con string "Plutus"))
   ).
\end{verbatim}
Note however that this syntax is not currently supported by most Plutus Core tools at the time of writing.



\paragraph{The $\ty{data}$ type.}
We provide a built-in type $\ty{data}$ which permits the encoding of simple data structures
for use as arguments to Plutus Core scripts.  This type is defined in Haskell as 
\begin{alltt}
data Data =
      Constr Integer [Data]
    | Map [(Data, Data)]
    | List [Data]
    | I Integer
    | B ByteString
\end{alltt}



\noindent In set-theoretic terms the denotation of $\ty{data}$ is
defined to be the least fixed point of the endofunctor $F$ on the category of
sets given by $F(X) = (\denote{\ty{integer}} \times X^*) \disj (X \times X)^* \disj
X^* \disj \denote{\ty{integer}} \disj \denote{\ty{bytestring}}$, so that
$$ \denote{\ty{data}} = (\denote{\ty{integer}} \times \denote{\ty{data}}^*)
               \disj (\denote{\ty{data}} \times \denote{\ty{data}})^*
               \disj \denote{\ty{data}}^*
               \disj \denote{\ty{integer}}
               \disj \denote{\ty{bytestring}}.
$$
We have injections
\begin{align*}
  \inj_C: \denote{\ty{integer}} \times \denote{\ty{data}}^* & \to \denote{\ty{data}} \\
  \inj_M: \denote{\ty{data}} \times \denote{\ty{data}}^*  & \to \denote{\ty{data}} \\
  \inj_L: \denote{\ty{data}}^* & \to \denote{\ty{data}} \\
  \inj_I: \denote{\ty{integer}} & \to \denote{\ty{data}} \\
  inj_B: \denote{\ty{bytestring}} & \to \denote{\ty{data}} \\
\end{align*}
\noindent and projections
\begin{align*}
  \proj_C: \denote{\ty{data}} & \to \withError{(\denote{\ty{integer}} \times \denote{\ty{data}}^*)}\\
  \proj_M: \denote{\ty{data}} & \to \withError{(\denote{\ty{data}} \times \denote{\ty{data}}^*)}\\
  \proj_L: \denote{\ty{data}} & \to \withError{\denote{\ty{data}}^* }\\
  \proj_I: \denote{\ty{data}} & \to \withError{\denote{\ty{integer}}}\\
  \proj_B: \denote{\ty{data}} & \to \withError{\denote{\ty{bytestring}} }\\
\end{align*}
\noindent which extract an object of the relevant type from a $\ty{data}$ object
$D$, returning $\errorX$ if $D$ does not lie in the expected component of the
disjoint union; also there are functions
$$
\is_C, \is_M, \is_L, \is_I, \is_B: \denote{\ty{data}} \to \denote{\ty{bool}}
$$
\noindent which determine whether a $\ty{data}$ value lies in the relevant component.


\subsection{Built-in functions}
The default set of built-in functions for the Alonzo release is shown in Figure~\ref{fig:alonzo-built-in-functions}.
\setlength{\LTleft}{-18mm}  % Shift the table left a bit to centre it on the page
\begin{longtable}[H]{|l|p{5cm}|p{5cm}|c|c|}
    \hline
    \text{Function} & \text{Signature} & \text{Denotation} & \text{Can} & \text{Note} \\
    & & & Fail? & \\
    \hline
    \endfirsthead
    \hline
    \text{Function} & \text{Type} & \text{Denotation} & \text{Can} & \text{Note}\\
    & & & Fail? & \\
    \hline
    \endhead
    \hline
    \caption{Built-in Functions}
    % This caption goes on every page of the table except the last.  Ideally it
    % would appear only on the first page and all the rest would say
    % (continued). Unfortunately it doesn't seem to be easy to do that in a
    % longtable.
    \endfoot
    \caption[]{Built-in Functions (continued)}
    \label{fig:alonzo-built-in-functions}
    \endlastfoot
    \T{addInteger}               & $[\ty{integer}, \ty{integer}] \to \ty{integer}$   & $+$ &  & \\
    \T{subtractInteger}          & $[\ty{integer}, \ty{integer}] \to \ty{integer}$   & $-$ &  & \\
    \T{multiplyInteger}          & $[\ty{integer}, \ty{integer}] \to \ty{integer}$   & $\times$ &  & \\
    \T{divideInteger}            & $[\ty{integer}, \ty{integer}] \to \ty{integer}$   & $\divfn$   & Yes & \ref{note:integer-division-functions}\\
    \T{modInteger}               & $[\ty{integer}, \ty{integer}] \to \ty{integer}$   & $\modfn$   & Yes & \ref{note:integer-division-functions}\\
    \T{quotientInteger}          & $[\ty{integer}, \ty{integer}] \to \ty{integer}$   & $\quotfn$  & Yes & \ref{note:integer-division-functions}\\
    \T{remainderInteger}         & $[\ty{integer}, \ty{integer}] \to \ty{integer}$   & $\remfn$   & Yes & \ref{note:integer-division-functions}\\
    \T{equalsInteger}            & $[\ty{integer}, \ty{integer}] \to \ty{bool}$      & $=$ &  & \\
    \T{lessThanInteger}          & $[\ty{integer}, \ty{integer}] \to \ty{bool}$      & $<$ &  & \\
    \T{lessThanEqualsInteger}    & $[\ty{integer}, \ty{integer}] \to \ty{bool}$      & $\leq$ &  & \\
    %% Some of the signatures look like $ ... $ \mbox{\;\; $ ... $} to allow a break with some indentation afterwards
    \T{appendByteString}         & $[\ty{bytestring}, \ty{bytestring}] $ \mbox{$\;\; \to \ty{bytestring}$}
                                           & $([c_1, \dots, c_m], [d_1, \ldots, d_n]) $ \mbox{$\;\; \mapsto [c_1,\ldots, c_m,d_1, \ldots, d_n]$} &  & \\
    \T{consByteString}         & $[\ty{integer}, \ty{bytestring}] $ \mbox{$\;\; \to \ty{bytestring}$}
                                          & $(c,[c_1,\ldots,c_n]) $ \mbox{$\;\;\mapsto [\text{mod}(c,256) ,c_1,\ldots,c_{n}]$} &  & \\
    \T{sliceByteString}        & $[\ty{integer}, \ty{integer}, \ty{bytestring]} $  \mbox {$\;\; \to  \ty{bytestring}$}
                                                   &   $(s,k,[c_0,\ldots,c_n])$ \mbox{$\;\;\mapsto [c_{\max(s,0)},\ldots,c_{\min(s+k-1,n-1)}]$}
                                                   &  & \ref{note:slicebytestring}\\
    \T{lengthOfByteString}       & $[\ty{bytestring}] \to \ty{integer}$ & $[] \mapsto 0, [c_1,\ldots, c_n] \mapsto n$ &  & \\
    \T{indexByteString}          & $[\ty{bytestring}, \ty{integer}] $ \mbox{$\;\; \to \ty{integer}$}
                                                   & $([c_0,\ldots,c_{n-1}],j)$ \mbox{$\;\;\mapsto
                                                       \left\{ \begin{array}{ll}
                                                         c_i & \mbox{if $0 \leq j \leq n-1$} \\
                                                         \errorX & \mbox{otherwise}
                                                       \end{array}\right.$} & Yes & \\
    \T{equalsByteString}         & $[\ty{bytestring}, \ty{bytestring}] $ \mbox{$\;\; \to \ty{bool}$}   & = &  & \ref{note:bytestring-comparison}\\
    \T{lessThanByteString}       & $[\ty{bytestring}, \ty{bytestring}] $ \mbox{$\;\; \to \ty{bool}$}   & $<$ &  & \ref{note:bytestring-comparison}\\
    \T{lessThanEqualsByteString} & $[\ty{bytestring}, \ty{bytestring}] $ \mbox{$\;\; \to \ty{bool}$}   & $\leq$ &  & \ref{note:bytestring-comparison}\\
    \T{appendString}             & $[\ty{string}, \ty{string}] \to \ty{string}$
                                         & $([u_1, \dots, u_m], [v_1, \ldots, v_n]) $ \mbox{$\;\; \mapsto [u_1,\ldots, u_m,v_1, \ldots, v_n]$} &  & \\
    \T{equalsString}             & $[\ty{string}, \ty{string}] \to \ty{bool}$           & = &  & \\
    \T{encodeUtf8}               & $[\ty{string}] \to \ty{bytestring}$      & Convert a $\ty{string}$ to a \texttt{byte\-string}. & &
                                                                                                               \ref{note:bytestring-encoding} \\
    \T{decodeUtf8}               & $[\ty{bytestring}] \to \ty{string}$      & Convert a $\ty{bytestring}$ to a $\ty{string}$. & Yes
                                                                                                                  & \ref{note:bytestring-encoding} \\
    \T{sha2\_256}                & $[\ty{bytestring}] \to \ty{bytestring}$  & Hash a $\ty{bytestring}$ using \T{SHA\-256}. &  & \\
    \T{sha3\_256}                & $[\ty{bytestring}] \to \ty{bytestring}$  & Hash a $\ty{bytestring}$ using \T{SHA3\-256}. &  & \\
    \T{blake2b\_256}             & $[\ty{bytestring}] \to \ty{bytestring}$  & Hash a $\ty{bytestring}$ using \T{Blake2B\-256}. &  & \\
    \T{verifySignature}          & $[\ty{bytestring}, \ty{bytestring}, $ \mbox{$\;\; \ty{bytestring}] \to \ty{bool}$}
                                                  & Verify the signature using \T{Ed25519}. &  Yes & \ref{note:signature-verification}\\
    \T{ifThenElse}               & $[\forall a_*, \ty{bool}, a_*, a_*] \to a_*$
                                                 & \mbox{$(\mathtt{true},t_1,t_2) \mapsto t_1$}
                                                 \mbox{$(\mathtt{false},t_1,t_2) \mapsto t_2$} & & \\
    \T{chooseUnit}               & $[\forall a_*, \ty{unit}, a_*] \to a_*$        & $((), t) \mapsto t$ & & \\
    \T{trace}                    & $[\forall a_*, \ty{string}, a_*] \to a_*$      & $ (s,t) \mapsto t$ &  & \ref{note:trace}\\
    \T{fstPair}                  & $[\forall a_\#, \forall b_\#, \pairOf{a_\#}{b_\#}] \to a_\#$       & $(x,y) \mapsto x$ && \\
    \T{sndPair}                  & $[\forall a_\#, \forall b_\#, \pairOf{a_\#}{b_\#}] \to b_\#$       & $(x,y) \mapsto y$ & & \\
    \T{chooseList}               & $[\forall a_\#, \forall b_*, \listOf{a_\#}, b_*, b_*] \to b_*$
                                              & \mbox{$([], t_1, t_2) \mapsto t_1$,} \mbox{$([x_1,\ldots,x_n],t_1,t_2) \mapsto t_2\ (n \geq 1)$}. & & \\
    \T{mkCons}                   & $[\forall a_\#, a_\#, \listOf{a_\#}] \to \listOf{a _\#}$  & $(x,[x_1,\ldots,x_n]) \mapsto [x,x_1,\ldots,x_n]$ & Yes & \\
    \T{headList}                 & $[\forall a_\#, \listOf{a_\#}] \to a_\#$               & $[]\mapsto \errorX, [x_1,x_2, \ldots, x_n] \mapsto x_1$ & Yes & \\
    \T{tailList}                 & $[\forall a_\#, \listOf{a_\#}] \to \listOf{a_\#}$
                                        &  \mbox{$[] \mapsto \errorX$,} \mbox{$ [x_1,x_2, \ldots, x_n] \mapsto [x_2, \ldots, x_n]$} & Yes & \\
    \T{nullList}                 & $[\forall a_\#, \listOf{a_\#}] \to \ty{bool}$            & $ [] \mapsto \T{true},
                                                                                                    [x_1,\ldots, x_n] \mapsto \T{false}$& & \\
    \T{chooseData}               & $[\forall a_*, \ty{data}, a_*, a_*, a_*, a_*, a_*] \to a_*$
    & $ (d,t_C, t_M, t_L, t_I, t_B) $
    \smallskip
    \newline  % The big \{ was abutting the text above
    \mbox{$\;\;\mapsto
                                                       \left\{ \begin{array}{ll}
                                                         t_C  & \mbox{if $\is_C(d)$} \\
                                                         t_M  & \mbox{if $\is_M(d)$} \\
                                                         t_L  & \mbox{if $\is_L(d)$} \\
                                                         t_I  & \mbox{if $\is_I(d)$} \\
                                                         t_B  & \mbox{if $\is_B(d)$} \\
                                                       \end{array}\right.$}  & & \\
    \T{constrData}               & $[\ty{integer}, \listOf{\ty{data}}] \to \ty{data}$          & $\inj_C$ & & \\
    \T{mapData}                  & $[\listOf{\pairOf{\ty{data}}{\ty{data}}}$ \mbox{$\;\; \to \ty{data}$}     & $\inj_M$& & \\
    \T{listData}                 & $[\listOf{\ty{data}}] \to \ty{data} $      & $\inj_L$& & \\
    \T{iData}                    & $[\ty{integer}] \to \ty{data} $            & $\inj_I$ & & \\
    \T{bData}                    & $[\ty{bytestring}] \to \ty{data} $         & $\inj_B$& & \\
    \T{unConstrData}             & $[\ty{data}] \to \pairOf{\ty{integer}}{\ty{data}} $            & $\proj_C$ & Yes& \\
    \T{unMapData}                & $[\ty{data}] $\mbox{$\;\; \to \listOf{\pairOf{\ty{data}}{\ty{data}}}$}  & $\proj_M$ & Yes& \\
    \T{unListData}               & $[\ty{data}] \to \listOf{\ty{data}} $                          & $\proj_L$ & Yes& \\
    \T{unIData}                  & $[\ty{data}] \to \ty{integer} $                                & $\proj_I$ & Yes& \\
    \T{unBData}                  & $[\ty{data}] \to \ty{bytestring} $                             & $\proj_B$ & Yes& \\
    \T{equalsData}               & $[\ty{data}, \ty{data}] \to \ty{bool} $                        & $ = $ & & \\
    \T{mkPairData}               & $[\ty{data}, \ty{data}]$ \mbox{\;\; $\to \pairOf{\ty{data}}{\ty{data}}$}  & $(x,y) \mapsto (x,y) $ & & \\
    \T{mkNilData}                & $[\ty{unit}] \to \listOf{\ty{data}} $                       & $() \mapsto []$ & & \\
    \T{mkNilPairData}            & $[\ty{unit}] $ \mbox{$\;\; \to \listOf{\pairOf{\ty{data}}{\ty{data}}} $}   & $() \mapsto []$ & & \\
    \hline 
\end{longtable}

\kwxm{Maybe try \texttt{tabulararray} to see what sort of output that gives for the big table.}

\note{Integer division functions.}
\label{note:integer-division-functions}
We provide four integer division functions: \texttt{divideInteger},
\texttt{modInteger}, \texttt{quotientInteger}, and \texttt{remainderInteger},
whose denotations are mathematical functions $\divfn, \modfn, \quotfn$, and
$\remfn$ which are modelled on the corresponding Haskell operations. Each of
these takes two arguments and will fail (returning $\errorX$) if the second one
is negative.  For all $a,b \in \Z$ with $a \ne 0$ we have
$$
\divfn(a,b) \times b + \modfn(a,b) = a
$$
$$
  |\modfn(a,b)| < |b|
$$\noindent and
$$
  \quotfn(a,b) \times b + \remfn(a,b) = a
$$
$$
  |\remfn(a,b)| < |b|.
$$
\noindent The $\divfn$ and $\modfn$ functions form a pair, as do $\quotfn$ and $\remfn$;
$\divfn$ should not be used in combination with $\modfn$, not should $\quotfn$ be used
with $\modfn$.

For positive divisors $b$, $\divfn$ truncates downwards and $\modfn$ always
returns a non-negative result ($0 \leq \modfn(a,b) \leq b-1$).  The $\quotfn$
function truncates towards zero.  Figure~\ref{fig:integer-division-signs} shows
how the signs of the outputs of the division functions depend on the signs of
the inputs ($+$ means non-negative, so includes 0).
\begin{table}[H]
  \centering
    \begin{tabular}{|cc|cc|cc|}
        \hline
        a & b & $\divfn$ & $\modfn$ & $\quotfn$ & $\remfn$ \\
        \hline
        $+$ & $+$ & $+$ & $+$ & $+$ & $+$ \\
        $-$ & $+$ & $-$ & $+$ & $-$ & $-$ \\
        $+$ & $-$ & $-$ & $+$ & $+$ & $+$ \\
        $-$ & $-$ & $+$ & $-$ & $+$ & $-$ \\
        \hline
        \end{tabular}
   \caption{Behaviour of integer division functions}
   \label{fig:integer-division-signs}
\end{table}
%% -------------------------------
%% |   n  d | div mod | quot rem |
%% |-----------------------------|
%% |  41  5 |  8   1  |   8   1  |
%% | -41  5 | -9   4  |  -8  -1  |
%% |  41 -5 | -9  -4  |  -8   1  |
%% | -41 -5 |  8  -1  |   8  -1  |
%% -------------------------------

\note{The \texttt{sliceByteString} function.}
\label{note:slicebytestring}
The application \texttt{[(builtin sliceByteString) (con integer $s$) (con
    integer $k$) (con bytestring $b$)]} returns the substring of $b$ of length
$k$ starting at position $s$; indexing is zero-based, so a call with $s=0$
returns a substring starting with the first element of $b$, $s=1$ returns a
substring starting with the second, and so on.  This function always succeeds,
even if the arguments are out of range: if $b=[c_0, \ldots, c_{n-1}]$ then the
  application above returns the substring $[c_i, \ldots, c_j]$ where
  $i=\max(s,0)$ and $j=\min(s+k-1, n-1)$; if $j<i$ then the empty string is returned.
  

\note{Comparisons of bytestrings.}
\label{note:bytestring-comparison}
Bytestrings are ordered lexicographically.  If we have $a = [c_1, \ldots, c_m]$
and $b = [d_1, \ldots, d_n]$ then
\begin{itemize}
\item $a = b$ if and only if $m=n$ and $c_i = d_i$ for $1 \leq i \leq m$
\item $a \leq b$ if $c_i = d_i$ for $1 \leq i \leq \min(m,n)$
\item $a<b$ if $a \leq b$ and $a \neq b$.
\end{itemize}
\noindent The empty bytestring is equal only to itself and is strictly less than all other bytestrings.

\kwxm{The lexicographic ordering means that $\mathtt{\#23456789} <
  \mathtt{\#24}$, which came as a slight surprise to me.  I think I was thinking
  of these as long numbers, not strings.}

\note{Encoding and decoding bytestrings.}
\label{note:bytestring-encoding}
The \texttt{encodeUtf8} and \texttt{decodeUtf8} functions convert between the
$\ty{string}$ type and the $\ty{bytestring}$ type.  We have defined
$\denote{\ty{string}}$ to consist of sequences of Unicode characters without
specifying any particular character representation, but
$\denote{\ty{bytestring}}$ consists of sequences of 8-bit bytes.  As the names
suggest, both functions use the well-known UTF-8 character encoding, where each
Unicode character is encoded using between one and four bytes: thus in general
neither function will preserve the length of an object; moreover, not all
sequences of bytes are valid representations of Unicode characters, and
\texttt{decodeUtf8} will fail if it receives an invalid input (but
\texttt{encodeUtf8} will always succeed).

\kwxm{In fact, strings are represented as sequences of UTF-16 characters, which
  use two or four bytes per character.  Do we need to mention that?  If we do,
  we'll need to be a little careful: there are sequences of 16-bit words that
  don't represent valid Unicode characters (for example, if the sequence uses
  surrogate codepoints improperly.  I don't think you can create a Haskell
  \texttt{Text} object (which is what our strings really are) that's invalid
  though.}


\note{Signature verification.}
\label{note:signature-verification}
The \texttt{verifySignature} performs cryptographic signature verification using
the Ed25519 scheme (\cite{ches-2011-24091}).  The function takes three bytestring
arguments: a public key $k$, a message $m$, and a signature $s$ (in that order).
The key $k$ must be exactly 32 bytes long and the signature $s$ must be exactly 64
bytes, and the function will fail if either is the wrong length; there is no
restriction on the length of the message. If $k$ and $s$ are the correct lengths
then \texttt{verifySignature} returns \texttt{true} if the private key
corresponding to $k$ was used to sign the message $m$ to produce $s$, otherwise
it returns \texttt{false}.

\note{The \texttt{trace} function.}
\label{note:trace}
An application \texttt{[(builtin trace) $s$ $v$]} ($s$ a \texttt{string}, $v$
any Plutus Core value) returns $v$.  We do not specify the semantics any
further.  An implementation may choose to discard $s$ or to perform some
side-effect such as writing it to a terminal or log file.

\kwxm{I sincerely hope that this is the only function we'll ever have that can
  do things like this.  I don't want to have to introduce some general notion of
  effectfulness into the notation to deal with the general case.}

\subsection{Cost accounting for built-in functions}
To follow.

\newpage
